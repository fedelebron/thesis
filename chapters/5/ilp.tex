\section{Integer Linear Programming}
\label{app:ilp}
Combinatorial optimization problems are frequently solved as either a linear programming (LP) problem, or an integer linear programming (ILP) problems. An LP problem is of the form

\begin{align*}
&\text{maximize }  & \langle c, x \rangle\\
&\text{subject to }& Ax &\le b&\\
&                  & x &\ge 0&\\
&                  & x &\in \R^n&
\end{align*}

\noindent where $A \in \R^{m \times n}$, and $b, c \in \R^m$.

Meanwhile, an ILP problem is of the form

\begin{align*}
&\text{maximize }  & \langle c, x \rangle\\
&\text{subject to }& Ax &\le b&\\
&                  & x &\ge 0&\\
&                  & x &\in \Z^n&
\end{align*}

\noindent where $A \in \Z^{m \times n}$, and $b, c \in \Z^m$.

Given an LP or ILP problem $Q$, we define $P(Q)$ as the convex hull of the set of $x$ that satisfy the constraints of the problem, regardless of whether or not $\langle c, x \rangle$ is maximized. Note this is a convex polyhedron in $\R^n$, regardless of whether $Q$ was an LP or an ILP.

Given an ILP $Q$, we can consider the LP that results from taking $Q$ and relaxing its integrality condition, that is, allowing $x$ to range over $\R^n$ instead of $\Z^n$. We call this LP $Q_\R$.

It is easily seen that for any ILP $Q$, we have $P(Q) \subseteq P(Q_\R)$.

It is well known that solving an LP can be done in polynomial time in $n$ and $m$\cite{lp}, for example using interior point methods or the ellipsoid method. Solving an ILP problem, however, is not known to be solvable in polynomial time. In fact, it is NP-complete to do so\cite{Garey:1990:CIG:574848}. However, due to the geometrical nature of this problem, one can often use tools from geometry in order to solve ILP instances very quickly.

Specifically, one has an ILP problem $Q$, and one takes advantage that $Q_\R$ is solvable in polynomial time, and is in general, "well-behaved". An optimal solution $x$ to $Q_\R$ need not be even valid for $Q$, but if is, then it is also optimal for $Q$. When it is not valid, one can try branching out on its fractional components. However, one can use information about previously visited nodes to avoid branching when it can't possibly yield a better solution than one we already have. Specifically, one uses the fact that maximizing the objective function over $Q_\R$ must always be as high or higher than maximizing the function over $Q$. Thus if one relaxes a subproblem and finds its objective value is less than the current maximum one knows, one need not branch any further into this subproblem.

The following pseudocode uses this idea:

\begin{codebox}
\Procname{$\proc{Branch-and-bound}(Q)$}
\li $L \gets \{Q_\R\}$
\li $x^* \gets \const{null}$
\li $v* \gets -\infty$
\li \While ($L \ne \emptyset$)
 \Do
\li    Let $l \in L$
\li    $L \gets L \setminus \{l\}$
\li    \If $l$ is infeasible
    \Do
\li      \Goto 4
    \End
\li    Let $x$ be a solution to $l$, with value $v$.
\li    \If $x$ is integral
    \Do
\li      \If $v > v^*$
         \Do
\li        $x^* \gets x$
\li        $v^* \gets v$
         \End
\li      \Goto 4
    \End
\li    \If $v > v^*$
    \Do
\li    Let $i$ be such that $x_i$ is fractional with value $v$.
\li    Let $\pi_1 = x_i \ge \lceil v \rceil$
\li    Let $\pi_2 = x_i \le \lfloor v \rfloor$.
\li    $L \gets L \cup \{l \cap \pi_1, l \cap \pi_2\}$
    \End
  \End

\li \Return $(x^*, v^*)$
\end{codebox}

This procedure is often used in practice, particularly by the CPLEX and SCIP solvers.

Another approach one can take is to attempt to eliminate fractional solutions by "cutting" the polytope $P(Q_\R)$: One adjoins further halfplanes to the formulation, removing $x$ from the new polytope, but not removing any valid integral solution. The goal is to trim $P(Q_\R)$ enough so that it resembles $P(Q)$, and such that a fast LP solver finds an integral optimum in $P(Q_\R)$, thus solving $Q$.

These cutting planes are often introduced as part of a technique known as "branch-and-cut". In this technique one uses the geometry of $Q_\R$ to help solve $Q$. This will be useful if one can "quickly" find such cutting halfplane, since otherwise the search for them would take too long to be useful. Generally one wants a polynomial algorithm to, given a fractional solution $x$, find a cutting halfplane that separates it from the polytope one is searching through. The procedure is as such:

\begin{codebox}
\Procname{$\proc{Branch-and-cut}(Q)$}
\li $L \gets \{Q_\R\}$
\li $x^* \gets \const{null}$
\li $v* \gets -\infty$
\li \While ($L \ne \emptyset$)
 \Do
\li    Let $l \in L$
\li    $L \gets L \setminus \{l\}$
\li    \If $l$ is infeasible
    \Do
\li      \Goto 4
    \End
\li    Let $x$ be a solution to $l$, with value $v$.
\li    \If $x$ is integral
    \Do
\li      \If $v > v^*$
         \Do
\li        $x^* \gets x$
\li        $v^* \gets v$
         \End
\li      \Goto 4
    \End
\li    \If $v > v^*$
    \Do
\li    \If there exists a cutting plane $\pi$ that separates $x$ from $l$
    \Do
\li      $l \gets l \cap \pi$.
\li      \Goto 7
    \End
\li    Let $i$ be such that $x_i$ is fractional with value $v$.
\li    Let $\pi_1 = x_i \ge \lceil v \rceil$
\li    Let $\pi_2 = x_i \le \lfloor v \rfloor$.
\li    $L \gets L \cup \{l \cap \pi_1, l \cap \pi_2\}$
    \End
  \End

\li \Return $(x^*, v^*)$
\end{codebox}

This procedure is not, in general, polynomial, nor is the branch-and-bound method. However, if the cuts $\pi$ we select to separate fractional solutions from $l$ are good, in that they trim away a lot of $l$, then the program will, in practice, finish quickly, because it will not need to branch on many variables. A large amount of study is dedicated to, for a particular problem, finding good families of halfplanes and fast algorithms to find separating halfplanes from these families.
