\chapter{Facet and vertex enumeration}
\label{app:facets}

In order to find facets for the polytope $Q$, we used two key pieces of software:
\begin{itemize}
\item zerOne\cite{Bussieck:1998:VSS:302316.302321}
\item PORTA\cite{PORTA}
\end{itemize}

zerOne is a vertex enumeration tool. It receives as input an LP file describing an integral polytope, and writes to standard output a list of every vertex. There is one caveat: The polytope \emph{must} must be contained in $\{0, 1\}^n$, for some n. Fortunately for us, this was indeed the case for the CSPAP formulation we presented. It is worth mentioning that the other tool, PORTA, also includes a way of enumerating integral vertices, but it is much slower, since it does not take advantage of the fact that every coordinate must be binary. In practice, zerOne was never the bottleneck, writing out millions of vertices in mere minutes. PORTA, on the other hand, would fail to produce answers when there were even a few tens of thousands of vertices.

PORTA does, however, implement a facet enumeration tool. PORTA itself is a suite of tools useful for integer linear programming. In particular, it implements the Fourier-Motzkin variable elimination method. For more information on how this is used to compute facets, see \cite{z-lop-95}.

The number of facets in a polytope generated by the CSPAP can be large. Thus, reading through every facet and trying to understand it will not be practical. Additionally, the chance equalities we mentioned will mean that often a facet will be expressed in a way only because a specific chance equality holds, which is non-obvious to spot in general.

Instead, what we did in order to find facets to add as cuts to our formulation was to take a small instance $I$, and find an optimal solution $v$ to $P_{\R}(I)$, the polytope with relaxed integrality constraints. If this optimal solution is integral, we try again with another instance. When we find a fractional optimal solution, we compute the facets of the integral polytope $P(I)$. We then go through each facet $f$ and test whether $v$ is on the correct side of $f$. For the violated facets, we take a closer look, and try to generalize that facet to a family of valid inequalities which would separate fractional solutions like $v$ from $P_{\R}(I)$. This will require considerable thought about the structure of the problem and why $v$ is violating it, since we will not find the answer purely as linear combinations of the formulation's constraints.

The program we used in order to do this is shown below. It expects as input PORTA's H-representation output, and a vertex file, with a single space-separated pair per line, indicating the variable, and its value in that vertex. Unmentioned variables are assumed to have a value of $0$.

\newpage

\inputminted[fontsize=\footnotesize]{haskell}{chapters/5/valid_facet.hs}

