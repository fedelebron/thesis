It is reasonable to expect that, in practice, one will not always need the full power of this model for a specific instance of the timetabling problem. Thus we classify instances by the number of courses they have, the number of weekly patterns each course has, and the number of possible start dates for each course. We derive polynomial bounds on the complexity of some cases, and prove that the other cases are in the class \npc. While this does not guarantee there are no polynomial algorithms which solve these latter cases, it is a good indicator that one should try other methods than the "standard" polynomial algorithms.

For the polynomial cases we develop models based on flow algorithms. This intuitively feels reasonable, since the problem we are solving is a very generalized version of a matching (where we match professors to classes), and matching problems are frequently solved using flow-based algorithms\cite{clrs}. For the families of instances which fall into the \npc class, we will later develop a model based on integer linear programming to solve them. The end result is that we have algorithms that solve these cases satisfactorily in practice.
