\section{Type-$(1, 1, n)$ instances}

\subsection{As an ad-hoc program}
This restriction is again in \textsc{P}, merely by a reduction to the single schedule case. Indeed, having fixed the rest of the input, and given $n$ possible schedules $p_1, \dots, p_n$, if $f(p_i)$ is the function that solves the problem for a single pattern $p_i$, then we can compute the solution to the multiple pattern version as simply
$$
\max_{1 \le i \le n} f(p_i)
$$

This is a polynomial number of calls to what we saw above was a function which runs in polynomial time in the input size, in particular $O(1)$. Thus this version also falls inside \textsc{P}.

\subsection{As a linear program}
This is, however, not particularly satisfying, since it does not immediately yield a formulation in terms of a known problem (that is, it is not a polynomial reduction in the sense of Karp). However, due to the polyhedral formulations for the single pattern case, we can obtain a polyhedral formulation for this reduction as well.

Consider the integer linear programming formulation of the single-pattern case, with the pattern fixed to $p_i$:

\begin{alignat*}{2}
  \text{maximize } & \langle q_i, x \rangle \\
  \text{subject to } & B_i x \le \mathbf{1}\\
                     & x \ge \mathbf{0}\\
                     & x \in \Z^M
\end{alignat*}

Call this problem $Q_i$.

We can construct the following integer linear program:
\begin{alignat*}{2}
  \text{maximize } & H \times \left(\sum_i \langle q_i, x \rangle\right) - y\\
  \text{subject to } & B_i x_i \le \mathbf{1}\ \forall\ i\\
                     & y \ge \langle q_i, x_i \rangle\\
                     & x_i \ge 0\\
                     & y \ge 0\\
                     & y \in \Z\\
                     & x_i \in \Z^M
\end{alignat*}

Where $H$ is some large constant. If we were to express this as a single matrix, we would obtain something like this:

\begin{align*}
A &= \begin{pmatrix}
B_1 & 0   & 0   & \dots & 0 & 0\\
0   & B_2 & 0   & \dots & 0 & 0\\
0   & 0   & B_3 & \dots & 0 & 0\\
\vdots & \vdots & \vdots & \dots & \vdots & \vdots\\
0 & 0 & 0 & \dots & B_n & 0\\
{q'}_1^t & 0 & 0 & \dots & 0 & -1\\
0 & {q'}_2^t & 0 & \dots & 0 & -1\\
0 & 0 & {q'}_3^t & \dots & 0 & -1\\
\vdots & \vdots & \vdots & \dots & \vdots & \vdots\\
0 & 0 & 0 & \dots & {q'}_n^t & -1\\
\end{pmatrix}\\
b &= (\mathbf{1}, \mathbf{1}, \mathbf{1}, \dots, \mathbf{1}, 0, 0, 0, \dots, 0)^t\\
c &= (0, 0, 0, \dots, 0, 1)
\end{align*}

And the integer linear program $Q$ could be expressed as
\begin{alignat}{2}
  \text{maximize } & H \times \langle c, x \rangle - y\\
  \text{subject to } & Ax \le b\\
                     & x \ge 0\\
                     & x \in \Z^N
\end{alignat}

\begin{prop}
$Q$ can be solved in polynomial time.
\end{prop}

\begin{proof}
Recall that by unimodularity the $Q_i$ had integral linear relaxations. We will not be able to show that $Q$ is totally unimodular, since this is not true in general, but what we will show is that the integral relaxation achieves the same value of the objective function.

Let $Q_\R$ be the linear relaxation of $Q$, and let $(x, y)$ be a solution of $Q_\R$. We want to show there exists a $(z, y) \in \Z^{NM+1}$ such that $\langle c, x \rangle = \langle c, z \rangle$, and $Az \le b$, with $z \ge 0$.

If $(x, y)$ is in $\Z^{NM+1}$, we are done. Suppose, then, that $(x, y)$ has at least one nonintegral coefficient. It cannot be the case that only $y$ is nonintegral, since if the other $x_i$ subvectors were all integral, and since the $q_i$ are all integral, we can minimize $y$ by setting it to be equal to $\max_i \langle q_i, x_i \rangle$. Thus, there exists at least one coefficient of $x$ which is not integral, and it belongs to some $x_i$ subvector.

Consider now that the $Q_i$, since their matrix is totally unimodular, have integral linear relaxations. So there exists an optimal solution $z_i$, such that $z_i \in \Z^M$. Now take $x'$ to be $x$, but with its $i$th subvector replaced by $z_i$. Clearly such a solution still is feasible for $B$, since the other subvectors make it valid for the other $B_j$, and $z_i$ makes it valid for $B_i$. Since $z_i$ is an optimal solution to $B_i$, it must be the case that $\langle q_i, x_i \rangle \le \langle q_i, z_i \rangle$, since $Q_i$ was a maximization problem. If this inequality were strict, we could replace $x$ by $x'$ in an optimal solution $(x', y)'$ of $Q_\R$, where $y' \ge y$ since the objective function of $Q_i$ might have increased. However, for large $H$ (specifically, $H > 1$) the decrease in the objective function due to $y$ will be dwarfed by the increase in the objective function of $Q_i$. This would yield a higher objective function, which is impossible since $x$ was an optimal solution to the relaxation $Q_\R$. Thus the inequality is actually an equality.

This means that $y$ is \emph{still} valid, since its only restriction was that $\forall j. y \ge \langle q_j, x_j \rangle$, and this is still true if we use the same $y$ but replace the $i$th subvector of $x$ by $z_i$. So $(x', y)$ is indeed a valid solution, and it has no nonintegral component in its $i$th subvector.

It is clear that we can do this for every single subvector in $x$, the optimal solution. Thus we can get a solution $z$, which has all its subvectors be integral, by replacing the nonintegral subvectors of $x$ by the integral solution vectors $z_i$. Clearly, then, the last coordinate of $z$ must be integral (since otherwise we could improve $y$, as before, by setting it to be the minimum over the $\langle q_i, z_i \rangle$), and thus $z$ is integral.

In all these transformations, we kept the value of $y$ constant and, in particular, integral. Thus $(z, y) \in \Z^{NM+1}$.

Thus $z$ is an integral optimum solution for $Q$, since it solves its linear relaxation and is integral, and thus if we want to solve $Q$ we can simply solve $Q_\R$ and the resulting optimal value will be optimal for $Q$ as well.
\end{proof}

Since the matrices involved all have constant size, and there are $n$ of them, the resulting linear programming instance has size polynomial in $n$, which is our input size. Thus we can solve these instances in polynomial time, via a reduction to linear programming.
