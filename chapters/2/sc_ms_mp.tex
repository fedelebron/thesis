\section{Type-$(1, n, m)$ instances}

These instances can also be solved in polynomial time. Conceptually, what is happening is we have a function $f(i, j)$ which takes constant time to find a solution for the type-$(1, 1, 1)$ instance where the start date has been fixed to the $i$th one, and the weekly schedule has been fixed to the $j$th one, and we are trying to compute

$$
\max_{\substack{1 \le i \le m\\1 \le j \le n}} f(i, j)
$$

By the same reasoning as before, we have a (relatively large, but still polynomially sized) block matrix, made of $B_i$ blocks, where the $B_i$ come from the type-$(1, 1, 1)$ linear programs, and we are attempting to find a maximum over all of those. For the same reasons as before, we can simply compute the answer in the linear relaxation and it will be valid in the integer version.

This concludes the analysis on single-course versions of the timetabling problem. All of the restrictions in this case were, as expected, solvable in polynomial time.
