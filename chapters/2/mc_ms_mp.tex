\section{Type-$(n, m, p)$ instances}

This, the general case, will clearly be in \nph since two subcases, namely type-$(n, m, 1)$ and type-$(n, 1, m)$ instances are in \nph.

For completeness' sake, seeing it's obvious that TIMETABLING is in NP, we can now state the following theorem

\begin{thm}
TIMETABLING is in \npc.
\end{thm}

In the next chapter, we will analyze the structure of an integer linear programming formulation for this general case.
