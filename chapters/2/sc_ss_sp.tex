\section{Type-$(1, 1, 1)$ instances}
We will show that this restriction falls squarely within the complexity class $\p$.

For convenience, since there is only one weekly schedule and starting week for the single course, we may consider a number $n'_{i, k}$ to be the number of professors of role $k$ that are needed on whichever class falls on the $i$th day, of the single course. Clearly this number may be computed in constant time, recalling that all the parameters we are analyzing the complexity in terms of are constants in this restriction.

Also for convenience, we will write $q_i$ for the quality the $i$th professor has in teaching the only course.

We will now see a few different ways of thinking about this reduction.
\subsection{As a maximum cost maximum flow problem}

One way to model these instances is as a maximum-cost maximum-flow problem. Specifically, we can create the following flow network, where capacities are in blue, and costs are in red:

\begin{center}
\begin{tikzpicture}
\node[v] (s) at (1, 1) {$s$};
\node[v] (p_j) at (3, 1) {$\alpha_j$};
\node (d1) at (3, 2) {$\vdots$};
\node (d2) at (3, 0) {$\vdots$};
%\node[v] (d3) at (3, 3) {$\alpha_1$};
%\node[v] (d4) at (3, -1) {$\alpha_p$};

\node[v] (p_jd_i) at (6, 1) {$\beta_{j, i}$};
\node (d5) at (6, 0) {$\vdots$};
\node (d6) at (6, 2) {$\vdots$};
%\node[v] (d7) at (6, 3) {$\beta_{1, 1}$};
%\node[v] (d8) at (6, -1) {$\beta_{|P|, d}$};

\node[v] (d_ir_k) at (9, 1) {$\gamma_{i, k}$};
\node (d9) at (9, 0) {$\vdots$};
\node (d10) at (9, 2) {$\vdots$};
%\node[v] (d11) at (9, 3) {$\gamma_{1, 1}$};
%\node[v] (d12) at (9, -1) {$\gamma_{d, m}$};

\node[v] (t) at (12, 1) {$t$};

\draw[->] (s) to node[above,blue] {$\infty$} node[below,red] {$q_j$} (p_j);

%\dottedarrows{{s}}{{d3, d4}};

\draw[->] (p_j) to node[above,blue] {$a_{j, i}$} node[below,red] {$0$} (p_jd_i);

\dottedarrows{{p_j}}{{p_jd_i}}{p_j}{p_jd_i};

\draw[->] (p_jd_i) to node[above,blue] {$r_{j, k}$} node[below,red] {$0$} (d_ir_k);

\dottedarrows{{p_jd_i}}{{d_ir_k}}{p_jd_i}{d_ir_k};

\draw[->] (d_ir_k) to node[above,blue] {$n'_{i, k}$} node[below,red] {$0$} (t);

%\dottedarrows{{d11, d12}}{{t}};

\end{tikzpicture}
\end{center}

Formally, the network is $G = (V, E)$, with

\begin{align*}
V = &\{s\} \cup \{t\}\\
&\cup \{\alpha_j \mid 1 \le j \le |P|\}\\
&\cup \{\beta_{j, i} \mid 1 \le j \le |P|, 1 \le i \le |D|\}\\
&\cup \{\gamma_{i, k} \mid 1 \le i \le |D|, 1 \le j \le |R|\}\\
E = &\{(s, \alpha_j)\ \forall\ j\}\\
    &\cup \{(\alpha_j, \beta_{j, i})\ \forall\ i, j \}\\
    &\cup \{(\beta_{j, i}, \gamma_{i, k})\ \forall\ j, i, k\}\\
    &\cup \{(\gamma_{i, k}, t)\ \forall\ i, k \}
\end{align*}

capacity function $c:E \to \mathbb{R} \cup \{\infty\}$, such that
\begin{align*}
c(s, \alpha_j) &= \infty\ \forall\ j\\
c(\alpha_j, \beta_{j, i}) &= a_{j, i}\ \forall\ i, j\\
c(\beta_{j, i}, \gamma_{i, k}) &= r_{j, k}\ \forall\ i, j, k\\
c(\gamma_{i, k}, t) &= n'_{i, k}\ \forall\ i, k
\end{align*}

and cost function $w:E \to \mathbb{R}$, such that

$$
w(e) =
\begin{cases}
q_j &\text{ if } e = (s, a_j) \text{ for some }j\\
0 & \text{otherwise}
\end{cases}
$$

We can think of the flow in this network as ``time'' dedicated by a professor at a given day, on a given role. A unit of flow goes from a professor to a given day (only if that professor is available that day), and to a given role (only if the professor could fit that role), and into fulfilling that day's quota for that role.

A professor cannot teach on an unavailable day, since flow can only go from the $j$th professor to the $i$th day if $a_{i, j} = 1$, and only one unit may go through in that case. This last fact ensures a professor cannot work twice in a single day, even under different roles.

He also cannot work in a role that is unavailable to him, since in order for flow to go from $\beta_{j, i}$ to $\gamma_{i, k}$, meaning ``the $j$th professor teaches on the $i$th day'' to ``the $i$th day has one $k$th role position covered'', unless $r_{j, k} = 1$, so one cannot consider an instance of that role covered for that day unless the professor from which that unit of flow came indeed can work in that role.

Lastly, the number of instances of the $k$th role that a given $\gamma_{i, k}$ needs for the $i$th day can not be greater than $n'_{i, k}$.

A flow $f$ in $G$ will be a valid assignment of professors exactly when $f(s, t) \ge \sum_{i, k} n'_{i, k}$. The constraints going into $t$ guarantee this implies $f(\gamma_{i, k}, t) = r_{i, k}\ \forall i, k$, which means each day's requirements of each role are met.

In such a solution, if a unit of flow goes from $s$, to $\alpha_j$, through $\beta_{j, i}$, then through $\gamma_{i, k}$, and finally to $t$, we consider that the $j$th professor teaches on the $i$th day with the $k$th role. The quality of this assignment is thus $q_j$.

Furthermore, to obtain an optimal solution, that is, a maximum sum of the qualities of the assignments, we need to maximize the cost of this flow, because this will mean the maximum value of the units going into (hence out of) the $\alpha_j$, which is the quality of the assignments we are making.

Since maximum-cost maximum-flow on $G$ is solvable in $O(poly(|V|, |E|))$ operations, and we can see our $|V|$ and $|E|$ are polynomials in $|P|, |D|, |R|$, we conclude this reduction can be solved in time $O(poly(|P|, |D|, |R|))$, and since we consider these to be constants, we have that $(1, 1, 1)$-TIMETABLING is in \textsc{P}.

\subsection{As an assignment problem}

Another way to model this restriction is as an assignment problem. For convenience, let us note $x = \sum_{i = 1}^{|P|} |A_i|$, $y = \sum_{k = 1}^{|S|} \sum_{l = 1}^{|R|} n'_{k, l}$, and $z = x - y$. We will see what these mean shortly. Now we create the following bipartite graph:

\begin{center}
\begin{tikzpicture}
%\node[v] (a11) at (0, 2) {$\alpha_{1, 1}$};
\node (d1) at (0, 1) {$\vdots$};
\node[v] (aij) at (0, 0) {$\alpha_{i, j}$};
\node (d2) at (0, -1) {$\vdots$};
%\node[v] (ant) at (0, -2) {$\alpha_{n, t}$};

%\node[v] (b111) at (3, 2) {$\beta_{1, 1, 1}$};
\node (d3) at (3, 1) {$\vdots$};
\node[v] (bklv) at (3, 0) {$\beta_{k, l, v}$};
\node (d4) at (3, -1) {$\vdots$};
%\node[v] (bcmw) at (3, -2) {$\beta_{c, m, n'_{c, m}}$};

\node[v] (h1) at (3, -2) {$h_1$};
\node (d5) at (3, -3) {$\vdots$};
\node[v] (hz) at (3, -4) {$h_z$};

\dottedarrows{{aij}}{{bklv, h1, hz}}{aij}{bklv};
\draw[->] (aij) to node[below, red] {if $l \in R_i$} node[above, blue] {$q_i$} (bklv);
\end{tikzpicture}
\end{center}

Formally, we have the graph $G = (V, E)$, such that
\begin{align*}
V = & \{\alpha_{i, j} \mid i \in P, j \in A_i\}\\
    & \cup \{\beta_{k, l, v} \mid k \in S, l \in R, 1 \le v \le n'_{k, l}\}\\
    & \cup \{h_i \mid 1 \le i \le z\}\\
E = & \{\{\alpha_{i, j}, \beta_{k, l, v}\}\ \forall\ i, j, k, l, v \mid l \in R_i\}\\
    & \cup \{\{\alpha_{i, j}, h_s\}\ \forall\ i, j, s\}
\end{align*}

with a cost function $w: E \to \mathbb{R}$, such that
\begin{align*}
w(\alpha_{i, j}, \beta_{k, l, v}) &= q_i\\
w(\alpha_{i, j}, h_s) &= 0
\end{align*}

Conceptually, $\alpha_{i, j}$ represents professor $i$ teaching on the $j$th day, and $\beta_{k, l, v}$ represents fulfilling the $v$th instance of the class which falls on the $k$th day's requirement for professors of role $l$. There are $n'_{k, l}$ ``copies'' of the pair $(k, l)$, as many as professors needed of the $l$th type. These are $\beta_{k, l, 1}, \dots, \beta_{k, l, n'_{k, l}}$. $x$ is the number of $\alpha$, and $y$ is the number of $\beta$. The $h$ nodes are added to the graph to make both partitions of the graph of the same size, thus allowing for a perfect matching.

Thus we can see that a maximum matching in this graph must attempt to assign professors to instances of role requirements. If indeed a matching has edges incident to every $\beta_{k, l, v}$, then if node $\alpha_{i, j}$ is its matched node, we say that professor $i$ is being assigned on day $j$ to cover role $l$, to the class which falls on the $k$th day. If an $\alpha_{i, j}$ is matched to some $h_s$, it means that the $i$th professor did not teach on the $j$th day (taking the $s$th ``vacation voucher'', if you will). We can see that the requirements of the assignment are satisfied, since

\begin{itemize}
\item No matching could assign a professor $i$ to teach two classes on day $j$ (since node $\alpha_{i, j}$ would be incident to two edges in the matching),
\item No matching could assign a professor $i$ to teach on a day $j$ in which he is not available, since $\alpha_{i, j} \in V$ only if $a_{i, j} = 1$,
\item If a matching has size $x$ ($= y + z$), it must have assigned an edge incident to every $\beta_{k, l, v}$ (indeed, to every node in the graph), thus we consider that every requirement of professors is satisfied for each class and each role, and
\item If such a matching exists, then its cost is the sum of the qualities of each professor chosen to teach (since edges to an $h_s$ have cost 0), which is the objective we want to maximize
\end{itemize}

Thus, we have reduced this problem to one of maximum-weight perfect matching in a bipartite graph, which is solvable in time $O(poly(|V|, |E|))$. In our case, $|V| \le |P||D|$, $|E| \le |S||R|(\max_{k, l} n'_{k, l})$. We can assume without loss of generality that $n'_{k, l} \le |P|\ \forall\ k, l$, since otherwise no possible solution exists. Thus $|E| \le |S||R||P|$, and this is a polynomial reduction, since these are constants in our analysis.

Moreover, given this formulation we have a very neat integer linear program. We note $v_1, \dots, v_N$ the nodes of $G$, and $e_1, \dots, e_M$ will be edges of the graph. For any edge $e$, $q_e$ will be $w(e)$, the weight of the edge $e$.

\begin{alignat*}{2}
  \text{maximize }   & q^t x \\
  \text{subject to } & \sum_{j=1}^M x_j \le 1, \ & 1 \le i \le N\\
                     & x_j \in \{1, 1\}, \ & 1 \le j \le M\\
\end{alignat*}

One thing we can remark is that $x_j \le 1$ is guaranteed by the first constraint, as any higher value would forcibly violate some constraint, namely the ones corresponding to each of the two incident vertices to that edge. Thus we can simply ask that $x_j \ge 0$

Additionally, if we consider $B \in \{0, 1\}^{N \times M}$ the incidence matrix of $G$, using the previous remark, we have the following integer linear program

\begin{alignat*}{2}
  \text{maximize } & q^t x \\
  \text{subject to } & Bx \le \mathbf{1}\\
                     & x \ge \mathbf{0}, x \in \mathbb{Z}^M
\end{alignat*}

Since $G$ is bipartite, $B$ is totally unimodular. Thus, by the Hoffman-Kruskal theorem, the linear program above is a convex function being maximized on an integral polyhedron, so we can compute its linear relaxation and optimize there, and the optimum solution will have the same value for the objective function. This shows another way in which to solve this reduction in polynomial time.
