\section{Type-$(1, 1, 1)$ instances}
We will show that this restriction falls within the complexity class $\p$.

For convenience, since there is only one weekly schedule and starting week for the single course, we may consider a number $n'_{i, k}$ and $N'_{i, k}$ to be the minimum and maximum number of professors of role $k$ that can be assigned to whichever class falls on the $i$th day, of the single course, or $0$ if no class falls on the $i$th day. Clearly this number may be computed in constant time, recalling that all the parameters we are analyzing the complexity in terms of are constants in this restriction.

Also for convenience, we will write $q_i$ for the quality the $i$th professor has in teaching the only course.

We will now see a few different ways of thinking about this reduction.
\subsection{As a maximum cost maximum flow problem}

One way to model these instances is as a maximum-cost maximum-flow problem. Before we introduce the general solution, we consider a small example scenario:

\begin{itemize}
  \item 1 week in the semester, and a 3 day week. The week is $\{$Monday, Tuesday, Wednesday$\}$.
  \item 3 professors: Mary, Jane, and Bob. Mary can work on Monday and Wednesday, Jane can work on Monday and Tuesday, and Bob can work on Tuesday and Wednesday.
  \item Mary can teach up to 5 classes, Bob can teach up to 2 classes, and Jane can teach up to 3 classes.
  \item There are 2 roles, lecturer and TA. Mary can act as a lecturer and a TA, Bob can act as a TA, and Jane can act as a lecturer.
  \item Mary's quality in this course is 6, Bob's is 4, and Jane's is 5.
  \item The course starts on Monday the $1$st, and it has $2$ classes. Its weekly pattern is $\{$Monday, Wednesday$\}$.
  \item The first class requires 1 lecturer, the second class requires 1 or 2 TAs.
\end{itemize}

We can think of a professor "flowing" "teaching units" through a network, and each class requires a certain number of these "teaching units" for each its roles. Thus we can create one node for each professor. These professors will let a certain amount of flow out, the maximum amount of which is given by the number of classes they can teach. These units of flow will go into a node representing "A professor who teaches with role $x$ on day $y$", and there will be a node for each day, which (since a single class falls on that day) will "require" a certain number of teaching units of each role.

A graphical example will hopefully make this clearer. In figure~\ref{fig:exampleflow} we show the construction for the above scenario, where we've omitted the capacity of an edge when it is $1$, and we omit the edge altogether when its capacity is $0$. We also omit its cost when it is $0$. We have colored  edge capacities in blue, and edge costs in red.

In order to find a solution for this instance, we run a maximum-cost maximum-flow algorithm on that graph, which yields a flow of $3$ with a cost of $16$, which is optimal. This is achieved by flowing $2$ units of flow through Mary, who will act as a lecturer on the first class and $TA$ on the second class, and $1$ unit of flow through Bob, who will act as a TA for the second class.

\begin{figure}
  \begin{tikzpicture}
    \node[box, fill=gray!20] (source) at (0, 1) {Source};
    \node[box, fill=blue!20] (Mary) at (3, 1) {Mary};
    \node[box, fill=blue!20] (Bob) at (3, -1) {Bob};
    \node[box, fill=blue!20] (Jane) at (3, 3) {Jane};
    \node[box, fill=red!20] (Mary1) at (5, 0) {Mary, M};
    \node[box, fill=red!20] (Mary2) at (5, 1) {Mary, T};
    \node[box, fill=red!20] (Mary3) at (5, 2) {Mary, W};
    \node[box, fill=red!20] (Bob1) at (5, -3) {Bob, M};
    \node[box, fill=red!20] (Bob2) at (5, -2) {Bob, T};
    \node[box, fill=red!20] (Bob3) at (5, -1) {Bob, W};
    \node[box, fill=red!20] (Jane1) at (5, 3) {Jane, M};
    \node[box, fill=red!20] (Jane2) at (5, 4) {Jane, T};
    \node[box, fill=red!20] (Jane3) at (5, 5) {Jane, W};
    \node[box, fill=green!20] (TA1) at (8, -2) {TA, M};
    \node[box, fill=green!20] (TA2) at (8, -1) {TA, T};
    \node[box, fill=green!20] (TA3) at (8, 0) {TA, W};
    \node[box, fill=green!20] (P1) at (8, 2) {Lecturer, M};
    \node[box, fill=green!20] (P2) at (8, 3) {Professor, T};
    \node[box, fill=green!20] (P3) at (8, 4) {Lecturer, W};
    \node[box, fill=gray!20] (sink) at (13, 1) {Sink};
    \draw[->] (source) to node[above,blue] {$5$} node[below,red] {$6$} (Mary);
    \draw[->] (source) to node[above,blue] {$2$} node[below,red] {$4$} (Bob);
    \draw[->] (source) to node[above,blue] {$3$} node[below,red] {$5$} (Jane);
    \draw[->] (Mary) to (Mary1);
    \draw[->] (Mary) to (Mary3);
    %\draw[->] (Mary) to node[above, blue] {$0$} (Mary2);
    \draw[->] (Jane) to (Jane1);
    \draw[->] (Jane) to (Jane2);
    %\draw[->] (Jane) to node[above, blue] {$0$} (Jane3);
    \draw[->] (Bob) to (Bob2);
    \draw[->] (Bob) to (Bob3);
    %\draw[->] (Bob) to node[below, blue] {$0$} (Bob1);
    \draw[->] (Mary1) to (TA1);
    \draw[->] (Mary1) to (P1);
    \draw[->] (Mary2) to (TA2);
    \draw[->] (Mary2) to (P2);
    \draw[->] (Mary3) to (TA3);
    \draw[->] (Mary3) to (P3);
    \draw[->] (Jane1) to (P1);
    \draw[->] (Jane2) to (P2);
    \draw[->] (Jane3) to (P3);
    \draw[->] (Bob1) to (TA1);
    \draw[->] (Bob2) to (TA2);
    \draw[->] (Bob3) to (TA3);
    \draw[->] (P1) to node[below, blue] {$(1, 1)$} (sink);
    %\draw[->] (TA1) to node[below, blue]  {$(0, 0)$} (sink);
    %\draw[->] (TA2) to node[below, blue]  {$(0, 0)$} (sink);
    %\draw[->] (P2) to node[below, blue]  {$(0, 0)$} (sink);
    %\draw[->] (P3) to node[above, blue] {$(0, 0)$} (sink);
    \draw[->] (TA3) to node[below, blue] {$(1, 2)$} (sink);
  \end{tikzpicture}
  \caption{Flow construction for the example type-$(1, 1, 1)$ scenario.}
  \label{fig:exampleflow}
\end{figure}



This example leads us to formulate the general version of this solution. We can create the flow network that appears in figure figure~\ref{fig:generalflow111}, where we say that the capacity of an edge is $i \in A_j$ when we mean that it is $1$ if $i \in A_j$, and $0$ otherwise. Similarly for $k \in R_j$.



\begin{figure}
  \begin{tikzpicture}
\node[v] (s) at (1, 1) {$s$};
\node[v] (p_j) at (3, 1) {$\alpha_j$};
\node (d1) at (3, 2) {$\vdots$};
\node (d2) at (3, 0) {$\vdots$};
%\node[v] (d3) at (3, 3) {$\alpha_1$};
%\node[v] (d4) at (3, -1) {$\alpha_p$};

\node[v] (p_jd_i) at (6, 1) {$\beta_{j, i}$};
\node (d5) at (6, 0) {$\vdots$};
\node (d6) at (6, 2) {$\vdots$};
%\node[v] (d7) at (6, 3) {$\beta_{1, 1}$};
%\node[v] (d8) at (6, -1) {$\beta_{|P|, d}$};

\node[v] (d_ir_k) at (9, 1) {$\gamma_{i, k}$};
\node (d9) at (9, 0) {$\vdots$};
\node (d10) at (9, 2) {$\vdots$};
%\node[v] (d11) at (9, 3) {$\gamma_{1, 1}$};
%\node[v] (d12) at (9, -1) {$\gamma_{d, m}$};

\node[v] (t) at (12, 1) {$t$};

\draw[->] (s) to node[above,blue] {$\max_j$} node[below,red] {$q_j$} (p_j);

%\dottedarrows{{s}}{{d3, d4}};

\draw[->] (p_j) to node[above,blue] {$i \in A_j$} node[below,red] {$0$} (p_jd_i);

\dottedarrows{{p_j}}{{p_jd_i}}{p_j}{p_jd_i};

\draw[->] (p_jd_i) to node[above,blue] {$k \in R_j$} node[below,red] {$0$} (d_ir_k);

\dottedarrows{{p_jd_i}}{{d_ir_k}}{p_jd_i}{d_ir_k};

\draw[->] (d_ir_k) to node[above,blue] {$(n'_{i, k}, N'_{i, k})$} node[below,red] {$0$} (t);

%\dottedarrows{{d11, d12}}{{t}};

\end{tikzpicture}
\caption{The general flow construction for type-$(1, 1, 1)$ instances.}
\label{fig:generalflow111}
\end{figure}

Formally, the network is $G = (V, E)$, with

\begin{align*}
V = &\{s\} \cup \{t\}\\
&\cup \{\alpha_j \mid j \in P\}\\
&\cup \{\beta_{j, i} \mid j \in P, i \in S\}\\
&\cup \{\gamma_{i, k} \mid i \in S, k \in R\}\\
E = &\{(s, \alpha_j)\ \forall\ j\}\\
    &\cup \{(\alpha_j, \beta_{j, i})\ \forall\ i, j \}\\
    &\cup \{(\beta_{j, i}, \gamma_{i, k})\ \forall\ j, i, k\}\\
    &\cup \{(\gamma_{i, k}, t)\ \forall\ i, k \}
\end{align*}

We also have a capacity function $c:E \to \left(\mathbb{R} \cup \{\infty\}\right)^2$, such that
\begin{align*}
  c(s, \alpha_j) &= (0, \textstyle\min_j)\ \forall\ j\\
  c(\alpha_j, \beta_{j, i}) &= (0, i \in A_j)\ \forall\ i, j\\
  c(\beta_{j, i}, \gamma_{i, k}) &= (0, k \in R_j)\ \forall\ i, j, k\\
  c(\gamma_{i, k}, t) &= (n'_{i, k}, N'_{i, k})\ \forall\ i, k
\end{align*}

where a capacity of $(a, b)$ means lower bound of $a$ units of flow, and upper bound of $b$ units of flow; and cost function $w:E \to \mathbb{R}$, such that

$$
w(e) =
\begin{cases}
q_j &\text{ if } e = (s, \alpha_j) \text{ for some }j\\
0 & \text{otherwise}
\end{cases}
$$

We can think of the flow in this network as "time" dedicated by a professor at a given day, on a given role. A unit of flow goes from a professor to a given day (only if that professor is available that day), and to a given role (only if the professor could fit that role), and into fulfilling that day's quota for that role.

A professor cannot teach on an unavailable day, since flow can only go from the $j$th professor to the $i$th day if $i \in A_j$, and only one unit may go through in that case. This last fact ensures a professor cannot work twice in a single day, even under different roles.

He also cannot work in a role that is unavailable to him, since in order for flow to go from $\beta_{j, i}$ to $\gamma_{i, k}$, meaning ``the $j$th professor teaches on the $i$th day'' to ``the $i$th day has one $k$th role position covered'', unless $k \in R_j$, so one cannot consider an instance of that role covered for that day unless the professor from which that unit of flow came indeed can work in that role.

Lastly, the number of instances of the $k$th role that a given $\gamma_{i, k}$ needs for the $i$th day can not be smaller than $n'_{i, k}$, or greater than $N'_{i, k}$.

A flow $f$ in $G$ will be a valid assignment of professors exactly when $f(s, t) \ge \sum_{i, k} n'_{i, k}$. The constraints going into $t$ guarantee this implies $n'_{i, k} \le f(\gamma_{i, k}, t) \le N'_{i, k}\ \forall i, k$, which means each day's requirements of each role are met.

In such a solution, if a unit of flow goes from $s$, to $\alpha_j$, through $\beta_{j, i}$, then through $\gamma_{i, k}$, and finally to $t$, we consider that the $j$th professor teaches on the $i$th day with the $k$th role. The quality of this assignment is thus $q_j$.

Furthermore, to obtain an optimal solution, that is, a maximum sum of the qualities of the assignments, we need to maximize the cost of this flow, because this will mean the maximum value of the units going into (hence out of) the $\alpha_j$, which is the quality of the assignments we are making.

Since maximum-cost maximum-flow on $G$ is solvable in $O(poly(|V|, |E|))$ operations, and we can see our $|V|$ and $|E|$ are polynomials in $|P|, |D|, |R|$, we conclude this reduction can be solved in time $O(poly(|P|, |D|, |R|))$, and since we consider these to be constants, we have that $(1, 1, 1)$-TIMETABLING is in \textsc{P}.

\subsection{As a linear program}
Moreover, this formulation introduces an associated linear programming formulation. Specifically, we can formulate any maximum-cost, integer maximum-flow problem with integer upper and lower bounds and integer edge costs as an ILP with an integral linear relaxation, following ~\cite{Ahuja93}.

Specifically, if we have a network $G = (V, E)$, with $V = \{v_1, \dots, v_N\}, E = \{e_1, \dots, e_M\}$, a weighting $w = (w_1, \dots, w_M) \in \Z^M$, and upper and lower bounds $u = (u_1, \dots, u_M) \in \Z^M, l = (l_1, \dots, l_M) \in \Z^M$, we can consider its incidence matrix $B \in \{0, 1\}^{M \times N}$. We consider then the following linear program:

\begin{alignat*}{2}
  \text{maximize }   & \langle c, x \rangle & \\
  \text{subject to } & \begin{pmatrix}
                         B\\
                        -B\\
                         I_M\\
                        -I_M
                       \end{pmatrix} x &\le (0^N, 0^N, u, -l)^t\\
                     & x \in \Z^M &
\end{alignat*}

Where $I_M$ is the $M \times M$ identity matrix.

This linear program's matrix is totally unimodular as proved in ~\cite{Ahuja93}, and so it has an integral linear relaxation. Thus by relaxing the integrality constraints and solving the resulting linear program in polynomial time in $N \times M$, we have another polynomial algorithm for this restriction. Note that $G$, by the previous section's considerations, has a size polynomial in our input size.

\subsection{As an assignment problem}

In the specific case where we have no limit on classes taught by a single professor, and we have $\min(c, n, r) = \max(c, n, r)$, we can model this version of CSPAP as an assignment problem. For convenience, let us note $x = \sum_{i = 1}^{|P|} |A_i|$, $y = \sum_{k = 1}^{|S|} \sum_{l = 1}^{|R|} n'_{k, l}$, and $z = x - y$. We will see what these mean shortly. Now we create the following bipartite graph:

\begin{center}
\begin{tikzpicture}
%\node[v] (a11) at (0, 2) {$\alpha_{1, 1}$};
\node (d1) at (0, 1) {$\vdots$};
\node[v] (aij) at (0, 0) {$\alpha_{i, j}$};
\node (d2) at (0, -1) {$\vdots$};
%\node[v] (ant) at (0, -2) {$\alpha_{n, t}$};

%\node[v] (b111) at (3, 2) {$\beta_{1, 1, 1}$};
\node (d3) at (3, 1) {$\vdots$};
\node[v] (bklv) at (3, 0) {$\beta_{k, l, v}$};
\node (d4) at (3, -1) {$\vdots$};
%\node[v] (bcmw) at (3, -2) {$\beta_{c, m, n'_{c, m}}$};

\node[v] (h1) at (3, -2) {$h_1$};
\node (d5) at (3, -3) {$\vdots$};
\node[v] (hz) at (3, -4) {$h_z$};

\dottedarrows{{aij}}{{bklv, h1, hz}}{aij}{bklv};
\draw[->] (aij) to node[below, red] {$q_i$} node[above, blue] {$l \in R_i$} (bklv);
\end{tikzpicture}
\end{center}

Formally, we have the graph $G = (V, E)$, such that
\begin{align*}
V = & \{\alpha_{i, j} \mid i \in P, j \in A_i\}\\
    & \cup \{\beta_{k, l, v} \mid k \in S, l \in R, 1 \le v \le n'_{k, l}\}\\
    & \cup \{h_i \mid 1 \le i \le z\}\\
E = & \{\{\alpha_{i, j}, \beta_{k, l, v}\}\ \forall\ i, j, k, l, v \mid l \in R_i\}\\
    & \cup \{\{\alpha_{i, j}, h_s\}\ \forall\ i, j, s\}
\end{align*}

with a cost function $w: E \to \mathbb{R}$, such that
\begin{align*}
w(\alpha_{i, j}, \beta_{k, l, v}) &= q_i\\
w(\alpha_{i, j}, h_s) &= 0
\end{align*}

Conceptually, $\alpha_{i, j}$ represents professor $i$ teaching on the $j$th day, and $\beta_{k, l, v}$ represents fulfilling the $v$th instance of the class which falls on the $k$th day's requirement for professors of role $l$. There are $n'_{k, l}$ ``copies'' of the pair $(k, l)$, as many as professors needed of the $l$th type. These are $\beta_{k, l, 1}, \dots, \beta_{k, l, n'_{k, l}}$. $x$ is the number of $\alpha$, and $y$ is the number of $\beta$. The $h$ nodes are added to the graph to make both partitions of the graph of the same size, thus allowing for a perfect matching.

Thus we can see that a maximum matching in this graph must attempt to assign professors to instances of role requirements. If indeed a matching has edges incident to every $\beta_{k, l, v}$, then if node $\alpha_{i, j}$ is its matched node, we say that professor $i$ is being assigned on day $j$ to cover role $l$, to the class which falls on the $k$th day. If an $\alpha_{i, j}$ is matched to some $h_s$, it means that the $i$th professor did not teach on the $j$th day (taking the $s$th "vacation voucher"). We can see that the requirements of the assignment are satisfied, since

\begin{itemize}
\item No matching could assign a professor $i$ to teach two classes on day $j$ (since node $\alpha_{i, j}$ would be incident to two edges in the matching),
\item No matching could assign a professor $i$ to teach on a day $j$ in which he is not available, since $\alpha_{i, j} \in V$ only if $a_{i, j} = 1$,
\item If a matching has size $x$ ($= y + z$), it must have assigned an edge incident to every $\beta_{k, l, v}$ (indeed, to every node in the graph), thus we consider that every requirement of professors is satisfied for each class and each role, and
\item If such a matching exists, then its cost is the sum of the qualities of each professor chosen to teach (since edges to an $h_s$ have cost 0), which is the objective we want to maximize
\end{itemize}

Thus, we have reduced this problem to one of maximum-weight perfect matching in a bipartite graph, which is solvable in time $O(poly(|V|, |E|))$. In our case, $|V| \le |P||D|$, $|E| \le |S||R|(\max_{k, l} n'_{k, l})$. We can assume without loss of generality that $n'_{k, l} \le |P|\ \forall\ k, l$, since otherwise no possible solution exists. Thus $|E| \le |S||R||P|$, and this is a polynomial reduction, since these are constants in our analysis.

