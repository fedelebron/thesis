\section{Notation}
For a given instance $\mathcal{T}$ of the timetabling problem, we will use the following notation:
\begin{itemize}
\item $C_{\mathcal{T}}$ a set of courses,
\item $P_{\mathcal{T}}$ a set of professors,
\item $T_{\mathcal{T}}$ a set of professor types,
\item $D_{\mathcal{T}}$ a set of days of the week,
\item $M_{\mathcal{T}} \subset 2^{D_\mathcal{T}}$ a set of weekly day patterns,
\item $S_{\mathcal{T}}$ a set of weeks in the semester,
\item $W_{\mathcal{T}} : C_{\mathcal{T}} \to 2^{S_\mathcal{T}}$ a function which assigns to each course a set of available starting weeks,
\item $n_\mathcal{T}: C_\mathcal{T} \to \Z_+$ a function which assigns each course the number of classes it has,
\item $N_\mathcal{T} \subset \mathbb{Z}$, defined as $N = \{1 \dots \max_{c \in C_\mathcal{T}} n_\mathcal{T}(c)\}$,
\item For every $c \in C_\mathcal{T}$, $n \in \{1 \dots n_\mathcal{T}(c)\}$, and $t \in T_\mathcal{T}$ we have $\max_\mathcal{T}(c, n, t)$ and $\min_\mathcal{T}(c, n, t): \Z+$, specifying the minimum and maximum number of professors of type $t$ required to teach the $n$th class of course $c$,
\item For every $c \in C_\mathcal{T}$, we have ${M_\mathcal{T}}_c \subset M_\mathcal{T}$, a set of allowed patterns for course $c$,
\item For every $p \in P_\mathcal{T}$, we have ${T_\mathcal{T}}_p \subset T_\mathcal{T}$, the set of allowed professor types for professor $p$,
\item For every $p \in P_\mathcal{T}$, we have a set ${A_\mathcal{T}}_p \subset S \mathcal{T}$,
\item For every $p \in P_\mathcal{T}$, $c \in C_\mathcal{T}$, we have a number ${q_\mathcal{T}}_{i, j} \in [0 \dots 7]$, indicating the quality of professor $p$ when teaching course $c$, and finally,
\item For every $p \in P_\mathcal{T}$, we have a number ${\max_{\mathcal{T}}}_p \in \Z+$, indicating the maximum number of classes professor $p$ is willing to teach
\end{itemize}

As is to be expected, when clear we will omit the subscript $\mathcal{T}$.
