\section{Timetabling problems}

Timetabling problems encompass several different types of problems. Classical problems include the course-teacher scheduling\cite{dewerra}, course and examination scheduling\cite{carterlaporte}, student scheduling\cite{studentsched}, and classroom scheduling\cite{classroom}. Due to their difficulty in practice, operations research has dedicated several volumes of text to the solution of these problems. Several approaches exist using integer linear programming models.

This work is motivated by a real case where a timetable design was required. Specifically, the "Construyendo Mis Sueños" program at the University of Chile was interested in planning their courses for professional training. Due to its volume and complexity, an automatic solution was required. An initial integer linear programming model\cite{PintoValdes} was proposed by Pinto Valdés, as part of a broader analysis of that particular program. We will analyze the problem from a theoretical standpoint, and provide an alternative integer linear programming model.

The unique aspects of this version of the problem are the variation in course start weeks and course weekly patterns. We will study how these affect the complexity of the problem.
