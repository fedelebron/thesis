\section{Formal definition}
An instance $\mathcal{T}$ of CSPAP consists of several pieces of data:

\begin{itemize}
\item $C$ a set of courses,
\item $P$ a set of professors,
\item $n: C \to \Z_+$ a function which assigns each course the number of classes it has
\item For every $p \in P$, $c \in C$, we have a number $q_{p, c} \in [0 \dots 7]$, indicating the quality of professor $p$ when teaching course $c$
\end{itemize}

Our goal will be to satisfy the requirements of each class in each course, and in doing so, attempt to maximize the overall quality of these assignments. To define the requirements, we need the following:
\begin{itemize}
\item $R$ a set of professor roles,
\item For every $c \in C$, $n \in \{1 \dots n(c)\}$, and $r \in R$ we have $\max(c, n, r)$ and $\min(c, n, r): \Z+$, specifying the minimum and maximum number of professors of role $r$ required to teach the $n$th class of course $c$,
\item For every $p \in P$, we have $R_p \subset R$, the set of allowed professor roles for professor $p$
\end{itemize}

Thus we need to assign professors to roles in classes, but only roles they are allowed. We must assign a minimum number of each one for every class, but since we'd want to maximize the sum of the qualities, we'd be tempted to assign all the professors we can. That's where the upper limit comes in, and restricts us.

Classes will be scheduled on certain days. When a class will be scheduled depends on its course's weekly pattern and starting week. Thus we need some more pieces of data:
\begin{itemize}
\item $WD$ a set of days of the week,
\item $M \subset 2^{WD}$ a set of weekly patterns,
\item $W$ a set of weeks in the semester,
\item For every $c \in C$, a set $W_c \subset 2^W$, the set of available starting weeks for $c$,
\item For every $c \in C$, we have $M_c \subset M$, a set of allowed schedules for course $c$,
\end{itemize}

We also have to take into account professor availability, given by the following pieces of data:
\begin{itemize}
\item For every $p \in P$, we have a set $A_p \subset W \times D$, the set days when professor $p$ is available,
\item For every $p \in P$, we have a number $\max_p \in \Z+$, indicating the maximum number of classes professor $p$ is willing to teach
\end{itemize}

Given an instance of CSPAP, we define $N \subset \mathbb{Z}$ as $N = \{1 \dots \max_{c \in C} n(c)\}$. We will also define $D = W \times WD$, the set of days in the semester.

Formally, a solution to a CSPAP instance is a triple $\mathcal{S} = (w, m, a)$, where $w: C \to W$ gives the starting week for each course, $m: C \to M$ gives the weekly schedule for each course, and $a: P \times D \to C \times N \times R \cup \{\bot\}$ says, for each $(p, d)$, what professor $p$ does on day $d$:
\begin{itemize}
\item If $a(p, d) = (c, n, r)$, then professor $p$ on day $d$ teaches the $n$th class of course $c$, as the $r$th role
\item If $a(p, d) = \bot$, then professor $p$ on day $d$ does not teach any courses
\end{itemize}

A solution must also meet the requirements specified in the instance. Formally, these are

\begin{itemize}
\item Every starting week is valid: $\forall\ c\ \in C. w(c) \in W(c)$.
\item Every weekly schedule is valid: $\forall\ c\ \in C. m(c) \in M_c$.
\item Every professor is available on days where he works: $a(p, d) \ne \bot \Rightarrow d \in A_p$.
\item Every professor can fulfil roles to which he is assigned: $a(p, d) = (c, n, r) \Rightarrow r \in R_p$.
\item No professor works more days than allowed: $\forall\ p\ \in P. \sum_{d \in D} \delta_{a(p, d, s) \ne \bot} \le \max_p$, where $\delta$ is the Kronecker delta.
\item The number of professors working on a given role on a given class are within the $\min$ and $\max$ parameters for that class: $\forall c \forall n \forall r. \min(c, n, r) \le \sum_{p \in P} \delta_{\exists d. a(p, d) = (c, n, r)} \le \max(c, n, r)$.
\end{itemize}

To a solution $\mathcal{S}$ we associate a value $V(\mathcal{S})$, defined as such:
\begin{align*}
V((w, m, a)) &= \sum_{\alpha \in a} V(\alpha)\\
V(\alpha) &= \sum_{\substack{p \in P \\d \in D}} Q(p, \alpha(p, d))\\
Q(p, x) &= \begin{cases}
q_{p, c} & \text{ if } x = (c, n, r)\\
0 &\text{ if } x = \bot
\end{cases}
\end{align*}

That is, the value of a solution is the sum of the qualities of each assignment of a professor to each class of a course. This is intuitively a good objective to maximize, since we know some professors are better at teaching some courses than others, and we would like to exploit this and assign each professor to the courses he or she is best suited for. Thus if we attempt to maximize the sum of the qualities of these assignments, we will attempt to assign each professor to courses to which he or she contributes a higher quality to.

Here's a full example, along with its optimal solution:

Let $C = \{\text{Biology}, \text{Computer Science}, \text{Mathematics}\}$, $P = \{\text{Mary}, \text{Bob}, \text{Jane}\}$, $n(\text{Biology}) = 2$, $n(\text{Computer Science}) = 3$, $n(\text{Mathematics}) = 3$. $q$ is defined as
\begin{align*}
  q(\text{Mary}, \text{Biology}) &= 3\\
  q(\text{Mary}, \text{Mathematics}) &= 2\\
  q(\text{Mary}, \text{Computer Science}) &= 8\\
  q(\text{Bob}, \text{Biology}) &= 4\\
  q(\text{Bob}, \text{Mathematics}) &= 4\\
  q(\text{Bob}, \text{Computer Science}) &= 7\\
  q(\text{Jane}, \text{Biology}) &= 3\\
  q(\text{Jane}, \text{Mathematics}) &= 5\\
  q(\text{Jane}, \text{Computer Science}) &= 8
\end{align*}

$R = \{\text{Full Professor}, \text{TA}, \text{Lab Assistant}\}$. Mary and Jane are full professors, and Mary can also act as a TA. Bob can act as a lab assistant and a TA.

Mary is available on Tuesdays, Wednesdays, and Fridays; Jane is available on Mondays, Wednesdays, Thursdays, and Fridays, except the 2nd Thursday of the semester; and Bob is available on Mondays, Thursdays, and Fridays. Mary can teach up to 5 classes, Jane can teach up to 6, and Bob can teach up to 6.

The requirements for each class (in this case, the lower bounds are the same as the upper bounds) are as follows:

\begin{itemize}
  \item For Biology
    \begin{itemize}
      \item Class 1: 1 TA, 1 professor
      \item Class 2: 1 lab assistant
    \end{itemize}
  \item For Mathematics
    \begin{itemize}
      \item Class 1: 1 professor
      \item Class 2: 1 TA, 1 lab assistant
      \item Class 3: 1 professor
    \end{itemize}
  \item For Computer Science
    \begin{itemize}
      \item Class 1: 1 professor
      \item Class 2: 1 TA
      \item Class 3: 1 lab assistant
    \end{itemize}
\end{itemize}

Biology has weekly patterns $\{$Monday, Tuesday, Wednesday$\}$ and $\{$Monday, Wednesday, Friday$\}$. Mathematics has weekly patterns $\{$Monday, Wednesday$\}$ and $\{$Thursday, Friday$\}$. Computer science has patterns $\{$Monday, Tuesday, Wednesday$\}$ and $\{$Tuesday, Thursday$\}$. The semester has 14 days, and starts on a Monday. Every course can start the first week or the second week of the semester. Thus $W = W_{\text{Biology}} = W_{\text{Mathematics}} = W_{\text{Computer Science}} = \{1, 8\}$, since those are the two possible start dates for each course: the start of the first week, or the start of the second week.


It is not particularly easy to see that an optimal solution to this instance is given by the triple $(w, m, a)$, where:

\begin{itemize}
  \item $w(\text{Biology}) = 1$
  \item $w(\text{Mathematics}) = 8$
  \item $w(\text{Computer Science}) = 8$
  \item $m(\text{Biology}) = \{\text{Monday}, \text{Tuesday}, \text{Wednesday}\}$
  \item $m(\text{Mathematics}) = \{\text{Thursday}, \text{Friday}\}$
  \item $m(\text{Computer Science}) = \{\text{Tuesday}, \text{Thursday}\}$
  \item $a(\text{Mary}, 12) = (\text{Mathematics}, 2, \text{TA})$
  \item $a(\text{Mary}, 9) = (\text{Computer Science}, 1, \text{Professor})$
  \item $a(\text{Jane}, 1) = (\text{Biology}, 1, \text{Professor})$
  \item $a(\text{Jane}, 8) = (\text{Mathematics}, 1, \text{Professor})$
  \item $a(\text{Jane}, 11) = (\text{Mathematics}, 3, \text{Professor})$
  \item $a(\text{Bob}, 1) = (\text{Biology}, 1, \text{TA})$
  \item $a(\text{Bob}, 5) = (\text{Computer Science}, 2, \text{TA})$
  \item $a(\text{Bob}, 11) = (\text{Computer Science}, 3, \text{Lab assistant})$
  \item $a(\text{Bob}, 12) = (\text{Mathematics}, 2, \text{Lab assistant})$
\end{itemize}

This assignment yields a value of $45$. Since it is impractical to try every possible combination of weekly pattern, starting week, and professor-class-role assignments, we wish to find efficient algorithms to solve these types of problems.\footnote{In particular, this instance was solved using the polyhedral model we will see in section 3, and its solving took 40 milliseconds on the SCIP software.}
\newpage
\subsection{Optimization and feasibility problems}

Given the above definitions, we can formulate the following optimization problem:
\decision{TIMETABLING-OPT}{An instance $\mathcal{T}$ of CSPAP.}{What is the largest $k$ such that there exists a solution $\mathcal{S}$ to $\mathcal{T}$, and $V(\mathcal{S}) = k$?}

We can also formulate the associated feasibility problem:
\decision{TIMETABLING}{An instance $\mathcal{T}$ of CSPAP.}{Does there exist a solution $\mathcal{S}$ to $\mathcal{T}$?}
