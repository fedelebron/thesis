\section{Dimension}

We begin by analyzing the dimension of the polytope $P(I)$ for an instance $I$ of CSPAP, which we'll call $\dim$. We see that we have six types of variables:
\begin{itemize}
\item $cp_{c, s}$, of which there are $|C| \times |M|$.
\item $csd_{c, w}$, of which there are $|C| \times |W|$.
\item $x_{c, l, p, k}$, of which there are $|C| \times N \times |P| \times |R|$.
\item $class\_date_{c, l, d}$, of which there are $|C| \times N \times |S|$.
\item $busy_{p, c, d}$, of which there are $|P| \times |C| \times |S|$.
\item $busy\_l_{p, c, l}$, of which there are $|P| \times |C| \times N$.
\end{itemize}

This means that the total number of variables is $|C| \times |M| + |C| \times |W| + |C| \times N \times |P| \times |R| + |C| \times N \times |S| + |P| \times |C| \times |S| + |P| \times |C| \times N$. After some simplification, this is $|C| \times |S| \times (|N| + |P|) + |C| \times (|P| \times (N + N \times |R|) +|M| + |W|)$. We will call this number $U$, since it will be an upper bound on the dimension of the polytope.


We find several linearly independent families of equalities which allow us to bound the dimension of the problem more tightly:

\begin{enumerate}
\item We have $|C|$ equations from \ref{eq:chosenstart}, each of which refer to different variables, and so are linearly independent and decrease our dimension upper bound by $|C|$.
\item From \ref{eq:chosenpattern} we obtain $|C|$ more linearly independent equations.
\item From \ref{eq:oneclassdate} we get $\sum_{i = 1}^{|C|} n(i)$ equalities, which again are linearly independent since they refer to distinct variables.
\item From \ref{eq:maxclassdates} we find $|S| \times (\sum_{i=1}^{|C|} (N - n(i))) = (\sum_{i=1}^{|C|} |S| \times N) - (\sum_{i=1}^{|C|} |S| \times n(i)) = |C| \times |S| \times N - |S| \times (\sum_{i = 1}^{|C|} n(i))$ equalities.
\item From \ref{eq:noinvalidclasses} we get an equality for every $(c, l, p, k)$ such that $l > n(c)$. There are $\sum_{i = 1}^{|C|} |P| \times |R| \times (N - n(i)) = |P| \times |R| \times |N| \times |C| - |P| \times |R| \times K$ of these.
\item While this is not a set of equations but of inequalities, from \ref{eq:availability} we get equalities for every $(p, d)$ such that $d \not\in A_p$. This gives us $\sum_{i=1}^{|P|} |S - A_i| = \sum_{i=1}^{|P|} |S| - |A_i| = |S| \times |P| - \sum_{i=1}^{|P|} |A_i|$ equalities.
\item In the same vein, from \ref{eq:rolevalidity} we get equalities for every $(c, l, p, k)$ such that $k \not\in R_p$. There are $\sum_{i = 1}^{|P|} |C| \times N \times (|R| - |R_i|) = |P| \times |C| \times N \times |R| - |C| \times N \times (\sum_{i=1}^{|P|} |R_i|)$ of these.
\item We have $|P| \times |C| \times N$ equalities stemming from the totally linear dependence of $busy\_l$ to the other variables. These are used simply for convenience, and so they add no dimensions to the polytope.
\end{enumerate}

For brevity's sake we'll denote $K = \sum_{i=1}^{|C|} n(i)$, $T = \sum_{i=1}^{|P|} |A_i|$, and $Q = \sum_{i=1}^{|P|} |R_i|$.

Considering these families of equalities, which are both linearly independent and linearly independent when combined, we can restrict $\dim$ by $U - 2|C| - (\sum_{i = 1}^{|C|} n(i)) - |C| \times |S| \times N + |S| \times (\sum_{i = 1}^{|C|} n(i)) - |P| \times |C| \times N$.

By noting $K = \sum_{i=1}^{|C|} n(i)$, we obtain the following result:

\begin{thm}
Let $\dim$ be the dimension of the polytope where there are $|C|$ courses, $|S|$ days in the semester, $|P|$ professors, $|R|$ roles, $|M|$ weekly schedules, $|W|$ starting weeks, $N$ is the maximum number of classes a course has, and $K$ is the sum of the number of classes of each course. Then
\begin{align*}
  \dim \le &|C| \times (N \times Q\\
                      &\hspace{1cm}+ P \times S\\
                      &\hspace{1cm}- N \times P \times R\\
                      &\hspace{1cm}+ W\\
                      &\hspace{1cm}+ M\\
                      &\hspace{1cm}- 2)\\
           &+ K \times (|P| \times |R| + |S| - 1)\\
           &- |P| \times |S|\\
           &+ T
\end{align*}
\end{thm}
\begin{proof}
We will show that the set of hyperplanes defined by the equations above are linearly independent. If the number of variables in the ILP formulation is $U$, and the number of equations is $\Sigma$, this will put an upper bound of $U - \Sigma$ on the dimension of the polytope associated with the ILP.

We will call the equations coming from the $i$th item, for $1 \le i \le 8$, $E_i$. First, let's show that each set $E_i$ is linearly independent. Since the $E_i$ do not share coefficients between different $i$, this will show that their union (of size $\Sigma$) is a linearly independent set.

\begin{itemize}
\item For $E_1$ we have $|C|$ equations. The $j$th equation here will be $\sum_{w \in W} csd_{j, w} = 1$. Then two equations $j$ and $j'$ in $E_1$ will not share any variables, since all of the first one's variables have first index $j$, and all of the second one's variables have first index $j'$.

\item For $E_2$ we have $|C|$ equalities, which are linearly independent for essentally the same reason as $E_1$ are.

\item For $E_3$ we have $K$ equations, but each one will have a unique pair of $(c, l)$ indices, so again there is no intersection among the variables used.

\item For $E_4$ and $E_5$  we see that the indices are again distinct for every equation.

\item For $E_6$ each equation has a unique pair of $(d, p)$ indices.

\item For $E_7$ each equation will have a unique 4-tuple $(c, l, p, k)$ of indices.

\item For $E_8$ clearly every $busy\_l$ variable has a unique index, and we have exactly one equation per variable.
\end{itemize}

Thus each set $E_1, \dots, E_8$ is linearly independent. There could only be a linear dependence between sets $E_3$ and $E_4$, and between $E_5$ and $E_7$, since they seem to refer to the same variables ($class\_date$ and $x$, respectively). However, in both cases the variables they work on are distinguished by having their $l$ coordinate be $\le n(c)$ or $> n(c)$, so the set of variables in these equations really is disjoint.

Thus every equation here is linearly independent, and the bound is obtained.
\end{proof}

We note that for the bound we obtain, the $|M|$ and $|W|$ variables are considered equally significant. This can be understood since in our ILP formulation they are treated essentially identically.

More conceptually, we can consider that every starting date we add forms, together with every weekly schedule, a new ``semester schedule'', which determines class dates for an entire semester.

\subsection{Chance hyperplanes}

One might hope that the bound given in the previous section is tight. Indeed, in simple cases we have seen it turns out this is actually the dimension of the polytope. For example, let us consider the following simple scenario:

\begin{itemize}
  \item A single course ($1$), with a single class ($1$), with a single possible weekly pattern ($1$) and possible starting date ($1$). The class requires $0$, $1$, or $2$ professors teaching with the only role there is ($1$).
  \item Two professors, $1$ and $2$, both able to fill the only role, and available on the day the class must happen.
\end{itemize}

Let the solution space be described by $7$-tuples of the form $(x_{1, 1, 1, 1}$, $x_{1, 1, 2, 1}$, $class\_date_{1, 1, 1}$, $cp_{1, 1}$, $csd_{1, 1}$, $busy_{1, 1, 1}$, $busy_{2, 1, 1})$. From \ref{eq:chosenstart} we have $csd_{1, 1} = 1$. From \ref{eq:chosenpattern} we obtain $cp_{1, 1} = 1$. From \ref{eq:oneclassdate} we know $class\_date_{1, 1, 1} = 1$. It turns out this is a minimal system for this instance. Namely, the integral solutions to this instance are exactly:

\begin{itemize}
\item (0, 0, 1, 1, 1, 0, 0)
\item (0, 0, 1, 1, 1, 0, 1)
\item (0, 0, 1, 1, 1, 1, 0)
\item (0, 0, 1, 1, 1, 1, 1)
\item (0, 1, 1, 1, 1, 0, 1)
\item (0, 1, 1, 1, 1, 1, 1)
\item (1, 0, 1, 1, 1, 1, 0)
\item (1, 0, 1, 1, 1, 1, 1)
\item (1, 1, 1, 1, 1, 1, 1)
\end{itemize}

Since for every other dimension than the central $3$ there's a vertex with that coordinate with a $0$ and another vertex with that coordinate with a $1$ in it, the dimension of the polyhedron is $7 - 3 = 4$. This is exactly the bound for this case, since we have

\begin{align*}
T = Q &= |P| = 2\\
|C| = N = |S| = |R| &= |W| = |M| = K = 1
\end{align*}

If we add the constraints in \ref{eq:noextrabusy} (which in the formulation we do not in order to simplify the solver's task, while not losing any information) we obtain two more equalities, namely

\begin{align*}
x_{1, 1, 1, 1} &= busy_{1, 1, 1}\\
x_{1, 1, 2, 1} &= busy_{2, 1, 1}
\end{align*}

Yielding the 4 solutions:

\begin{itemize}
\item (0, 0, 1, 1, 1, 0, 0)
\item (0, 1, 1, 1, 1, 0, 1)
\item (1, 0, 1, 1, 1, 1, 0)
\item (1, 1, 1, 1, 1, 1, 1)
\end{itemize}

These being the 4 possible scenarios: Either nobody teaches the class (which is fine given the constraints), $1$ teaches it, $2$ teaches it, or they both do. We notice that the vertices form a 2-cube in ${\Z/2\Z}^7$, and thus their dimension is $2$.

However, in general, the bound is unfortunately \emph{not} tight, due to what we will call "chance hyperplanes", or "chance equalities".

Take, for example, a course $c$ with a class $i$ such that $i$ requires at least 1 instance of a professor with role $j$. Furthermore, assume there is exactly one professor, the $k$th one, who can fulfill that role. Then we will have the following relevant constraints on the polytope:

\begin{align*}
\forall p \in P. x_{c, i, p, j} &\le r_{p, j}\\
1 &\le \sum_{p \in P} x_{c, i, p, j}\\
\forall p \in P. 0 &\le x_{c, i, p, j} \le 1
\end{align*}

Since only professor $k$ can fulfill role $j$, $r_{p, j} = 0 \forall p \ne k$. Thus we have

\begin{align*}
\forall p \in P \setminus \{k\} & x_{c, i, p, j} = 0
\end{align*}

Since these variables vanish, we conclude the following

\begin{align*}
1 &\le \sum_{p \in P \setminus \{k\}} x_{c, i, p, j} + x_{c, i, k, j}\\
  &= 0 + x_{c, i, k, j}\\
  &= x_{c, i, k, j} \le 1
\end{align*}

Thus we have a valid equality $x_{c, i, k, j} = 1$, which is not deduced solely from the problem's ILP, but depends on the actual data that each case has (in this case, on the fact that only professor $k$ can fulfill role $j$).

Another case arises when the upper ($\max(c, l, k)$) and lower ($\min(c, l, k)$) bounds for the number of professors $p$ of a given role $k$ in a given class $c$ coincide. These then give rise to an equality

$$
\sum_{p \in P} x_{c, l, p, k} = \min(c, l, k) = \max(c, l, k)
$$

A particular case of this is when $\max(c, l, k) = 0$, where we in fact get $|P|$ such equalities:

$$
\forall p \in P. x_{c, l, p, k} = 0
$$

It is for this reason that the dimension of the polytope will be, in general, lower than the above bound.
