\section{Valid inequalities}

In order to help an ILP solver find an optimal solution to our formulation we develop the following families of valid inequalities, meant as part of a branch-and-cut algorithm.

Given what we know about chance inequalities, the dimension of the polytope will not be easy to compute, in general. Thus whether or not these inequalities are facet-defining is something we will not be able to prove, since the dimension of the polytope itself will be unknown. However, these inequalities were all obtained by looking at small instances, doing vertex enumeration with the zerOne program\cite{Bussieck:1998:VSS:302316.302321}, doing facet enumeration via the Fourier-Motzkin transform with the PORTA program\cite{PORTA}, and extrapolating from the instance facets to generally valid families of inequalities. Thus at least in those cases, some of the inequalities were facet-defining by construction, and we suspect that they are strong, in general. For more information on the facet and vertex enumeration procedures, see appendix \ref{app:facets}.

%\input{busy_cuts}
\subsection{Class date cuts}

It is often the case that the instance data implies that no two classes of the same course are to be scheduled at the same time. In this case, we have a simple family of cuts. Consider the following situation:

\begin{itemize}
  \item Schedule 1: One class per week, on mondays.
  \item Schedule 2: Three classes per week, on mondays, wednesdays, and fridays.
\end{itemize}

Course $c$ has 3 classes, and the semester lasts one week (to simplify the scenario). It also has a single possible starting week, the first one. Assume there are enough professors available to fill all role requirements of course $c$. In this case, the only possible schedule $c$ can pick is schedule 2, since picking schedule $1$ would result in not scheduling two of $c$'s classes.

Now consider the linear relaxation of this scenario. It is perfectly possible for one to have $cp_{c, 1} = 0.5$, and $cp_{c, 2} = 0.5$. Considering the start week as $1$ (thus $csd_{c, 1} = 1$), the first monday as $1$, the problem constraints mean that

$$
(cp_{c, 1} - 1) + (csd_{c, 1} - 1) \le class\_date_{c, 1, 1}
$$

Since we said there was only one starting week, so $csd_{c, 1} = 1$. We also said $cp_{c, 1} = 0.5$. Thus our constraint becomes

$$
0.5 \le class\_date_{c, 1, 1}
$$

And we also said schedule $2$, with starting week $1$,  placed the $1$st class on day $1$

$$
0.5 \le class\_date_{_c, 1, 1}
$$

Now since every class must be allocated to some day, we have a constraint $\sum_{d \in D} class\_date_{c, 1, d} = 1$. So let's say our solution vector places $class\_date_{c, 1, 1} = 1$.

Since schedule $2$ has no day allocated for class $3$, thus we have a constraint

$$
(1 - cp_{c, 1}) + (1 - csd_{c, 1}) \ge class\_date_{c, 2, 1}
$$

Since $ds_{1, 1, 3, 1} = 0$. That is, picking the schedule $1$ and starting week $1$, the $3$rd class of a course does \emph{not} fall on the $1$st day. However, since we have a fractional $cp_{c, 1} = 0.5$, this becomes

$$
0.5 \ge class\_date_{c, 2, 1}
$$

So we can end up with a solution vector where $class\_date_{c, 2, 1} = 0.5$, and $class\_date_{c, 1, 1} = 1$. But then, if we sum for all $l$ the value of $class\_date_{c, l, 1}$ we get a value exceeding $1$, which is impossible since we must at most have one class per day for this course.

Thus we derive the following cut:

\begin{align}
&\forall c \in C,\\
&\forall d \in D,\\
&\sum_{l = 1}^{n(c)} class\_date_{c, l, d} \le 1
\end{align}

Thus rendering this fractional solution invalid, but still being a valid inequality for all integral solutions.

\subsection{Busy cuts}

Consider the following scenario:
\begin{itemize}
\item A single course, 1.
\item A single class, 1.
\item Two days, 1 and 2.
\item Two professors, available both days.
\item A single role, 1, which both professors can take.
\item A single starting week, 1.
\item Two weekly patterns, $p1$ and $p2$. $p1$ schedules class $1$ on day $1$, $p2$ schedules class $1$ on day $2$.
\end{itemize}

Then we could have the following fractional vertex:

\begin{itemize}
\item $csd_{1, 1} = 1$, remembering $csd_{c, sd}$ means course $c$ chooses starting week $sd$
\item $cp_{1, 1} = 0.5$, remembering $cp_{c, p}$ means course $c$ chooses schedule $p$
\item $cp_{1, 2} = 0.5$
\item $class\_date_{1, 1, 1} = 0.5$, remembering $class\_date_{c, l, d}$ means the $l$th class of course $c$ is scheduled on day $d$
\item $class\_date_{1, 1, 2} = 0.5$
\item $busy_{1, 1, 1} = 0$, remembering $busy_{p, c, d}$ means professor $p$ teaches a class for course $c$ on day $d$
\item $busy_{1, 1, 2} = 0$
\item $busy_{2, 1, 1} = 0$
\item $busy_{2, 1, 2} = 0$
\item $x_{1, 1, 1, 1} = 0.5$, remembering $x_{c, l, p, k}$ means professor $p$ teaches the $l$th class of course $c$ with role $k$
\item $x_{1, 1, 2, 1} = 0.5$
\end{itemize}

It may seem odd that we have class $1$ falling on days $1$ or $2$ (as per $class\_date$), and we've assigned professors $1$ and $2$ to teach the $1$st class, and yet \emph{neither} of them are busy on days $1$ or $2$!

But this makes sense given the constraints. Remember the way we lower bound $busy$:

$$
busy_{p, c, d} \ge busy\_l_{p, c, l} + class\_date_{c, l, d} - 1
$$

where $busy\_l_{p, c, l} = \sum_{k \in R} x_{c, l, p, k}$. In our case, with $d = 1$ or $d = 2$, we get $busy\_l_{1, 1, 1} = 0.5$, and $class\_date_{1, 1, d} = 0.5$. Thus the constraint is the same as

$$
busy_{1, 1, d} \ge 0
$$

which is trivial.

The problen here is that since each professor only works "half" the class, the class requirements are satisfied, but not the $busy$ requirements for each of the professors (at least conceptually - clearly the linear constraints themselves are satisfied!). 

We eliminate these vertices by noting that if a professor $p$ teaches the $l$th class of course $c$ with some role, and class $l$ can only fall on a subset $X$ of days, then it must hold that

$$
busy\_l_{p, c, l} \le \sum_{d \in X} busy_{p, c, d}
$$

Furthermore, we can always compute at least one such subset $X$: The set $d$ such that there's at least one valid weekly schedule and one valid starting week, such that the $l$th class falls on the $d$th day. This results in the following valid cutting plane:

\begin{align*}
&\forall c \in C\\
&\forall p \in P\\
&\forall 1 \le l \le n(c)\\
&busy\_l_{p, c, l} \ge \sum_{\substack{d \in D \\ s \in M\ \mid\ pp_{c, s} = 1\\ w \in W\ \mid\ psd_{c, w} = 1 \\ ds_{s, w, l, d} = 1}} busy_{p, c, d}
\end{align*}

The above fractional solution is cut, since this inequality says, for $c = 1, p = 1, l = 1$:

$$
busy\_l_{1, 1, 1} = \sum_{k \in R} x_{1, 1, 1, k} = x_{1, 1, 1, 1} \le busy_{1, 1, 1} + busy_{1, 1, 2}
$$

Where the right hand side is $0$, and the left hand side is $0.5$, rendering this solution invalid.

In practice, we see this family of cuts helps in a branch \& cut algorithm used in CPLEX and SCIP, lowering the solving time by about 10\% on large instances. As an improvement for a branch \& cut strategy, when one wishes to add a cut to a node in the branch \& cut tree, one can drop the terms corresponding to the schedules and starting days that haven't been explicitly zeroed out in the current node by the branching, tightening the bound. We note that this cut is a \emph{local} cut, since it need not be valid for other nodes in the tree, which might have chosen a different weekly schedule and starting date, and thus we cannot remove those terms from the summation.

\subsection{Busy subset cuts}
There is a particularly interesting extension of the previous family of cuts. Consider the following scenario:
\begin{itemize}
\item A single course, $1$.
\item The course has two classes, $1$ and $2$.
\item There's only one role, $1$ and each class requires $1$ professor with that role.
\item Two available weekly patterns, $1$ and $2$. Pattern $1$ places class $1$ on day $1$, and class $2$ on day $2$. Pattern $2$ places class $1$ on day $2$, and class $2$ on day $3$.
\item A single starting date, $1$.
\item Two professors, $1$ and $2$, available both days, and able to fill role $1$.
\end{itemize}

Thus we have $ds_{1, 1, 1, 1} = 1$, $ds_{1, 1, 2, 2} = 1$, $ds_{2, 1, 1, 2} = 1$, and $ds_{2, 1, 2, 3} = 1$, remembering that $ds_{p, sd, l, d} = 1$ means schedule $p$ with starting date $sd$ has the $l$th class fall on day $d$.

Thus we see that classes $\{1, 2\}$ must fall in either days $\{1, 2\}$ or $\{2, 3\}$. Consider, however, the following vertex:

\begin{itemize}
\item $cp_{1, 1} = 0.5$
\item $cp_{1, 2} = 0.5$
\item $class\_date_{1, 1, 1} = 0.5$
\item $class\_date_{1, 1, 2} = 0.5$
\item $class\_date_{1, 2, 2} = 0.5$
\item $class\_date_{1, 2, 3} = 0.5$
\item $x_{1, 1, 3, 1} = 0.5$
\item $x_{1, 1, 1, 1} = 0.5$
%\item $x_{1, 1, 2, 1} = 0.5$
\item $x_{1, 2, 1, 1} = 0.5$
\item $x_{1, 2, 2, 1} = 0.5$
%\item $busy_{1, 1, 1} = 0.5$
\item $busy_{1, 1, 2} = 0.5$
\item $busy_{2, 1, 2} = 0.5$
\item $busy_{3, 1, 1} = 0.5$
%\item $busy_{2, 1, 1} = 0$
%\item $busy_{2, 1, 3} = 0$
\end{itemize}

\noindent where all other variables are $0$. One may wonder how it is possible that $busy_{2, 1, 3} = busy_{1, 1, 3} =  0$, when $class\_date_{1, 2, 3} = 0.5$, and $x_{1, 2, 1, 1} = x_{1, 2, 2, 1} = 0.5$. That is, they (professors $1$ and $2$) are both "semi-responsible" for class $2$, and class $2$ semi-falls on day $3$, yet \emph{neither} of them works on day $3$!

The model's constraints aren't violated in this fractional solution, as this is our lower bound constraint on $busy$:

$$
busy_{p, c, d} \ge busy\_l_{p, c, l} + class\_date_{c, l, d} - 1
$$

for $p \in P$, $c \in C$, $1 \le l \le n(c)$, and $d \in D$.

Instantiated at $p \in \{1, 2\},$ $c = 1$, $l \in \{1, 2\}$, $d = 3$, and remembering that since we have only one role, $busy\_l_{p, c, l} = x_{c, l, p, 1}$, the model's busy lower bound constraints say:

\begin{itemize}
\item $busy_{1, 1, 1} \ge x_{1, 1, 1, 1} + class\_date_{1, 1, 3} - 1 \iff 0 \ge 0.5 + 0 - 1$ \greencheck
\item $busy_{1, 1, 2} \ge x_{1, 2, 1, 1} + class\_date_{1, 2, 3} - 1 \iff 0.5 \ge 0.5 + 0.5 - 1$ \greencheck
\item $busy_{2, 1, 1} \ge x_{1, 1, 2, 1} + class\_date_{1, 1, 3} - 1 \iff 0 \ge 0 + 0 - 1$ \greencheck
\item $busy_{2, 1, 2} \ge x_{1, 2, 2, 1} + class\_date_{1, 2, 3} - 1 \iff 0.5 \ge 0.5 + 0.5 - 1$ \greencheck
\end{itemize}

The previous family of cuts, the busy cuts, doesn't separate this vertex either, since we do have 
\begin{itemize}
\item $x_{1, 1, 1, 1} \le busy_{1, 1, 1} + busy_{1, 1, 2}$ \greencheck
\item $x_{1, 2, 1, 1} \le busy_{1, 1, 2} + busy_{1, 1, 3}$ \greencheck
\item $x_{1, 1, 2, 1} \le busy_{2, 1, 1} + busy_{2, 1, 2}$ \greencheck
\item $x_{1, 2, 2, 1} \le busy_{2, 1, 2} + busy_{2, 1, 3}$ \greencheck
\end{itemize}

What is missing, however, is that if professor $p$ works on a set $X$ of classes for course $c$, and the set of possible dates for those classes (i.e., the union, for all possible choices of schedule and starting date, of the days the classes in $X$ fall) is called $Y$, then the sum of the $busy\_l_{p, c, x}$ for $x \in X$ \emph{must} be smaller than or equal to the sum of the days, out of $Y$, that $p$ works on course $c$, because for every class $p$ works on, he works on exactly one day for that course, and we sum over all possible days these classes could fall in. Formally:

\begin{align*}
&\forall c \in C,\\
&\forall p \in P,\\
&\forall X \subset \{1, \dots, n(c)\}\\
&\text{Let }Y = \{d \in D \mid \exists l \in X, p \in M, w \in W \mid ds_{p, w, l, d} = 1\}\\
&\text{Then, }\\
&\sum_{l \in X} busy\_l_{p, c, l} \le \sum_{d \in Y} busy_{p, c, d}
\end{align*}

If we instantiate this family with $c = 1, p = 1, X = \{1, 2\}$, then we have $Y = \{1, 2, 3\}$, and the inequality

\begin{align*}
&x_{1, 1, 1, 1}& &+& &x_{1, 2, 1, 1}& &\le& &busy_{1, 1, 1}& &+& &busy_{1, 1, 2}& &+& &busy_{1, 1, 3}&\\
&0.5& &+& &0.5& &\le& &0& &+& &0.5& &+& &0&
\end{align*}

So this fractional solution violates this cut. We call this family of cuts the "busy subset" cuts. Conceptually, they are stronger than sums of the previous "busy" cuts, since if two pairs (weekly schedule, starting date) place two classes in $X$ on the same day $d$, these cuts only count that day once in the right hand side (since $d$ will only be counted once in $Y$), whereas summing over all the busy cuts would count them twice. Thus this is a tighter upper bound on $busy\_l$.

In practice, we observe that for small and medium sized instances, adding this family of cuts yields integral solutions for the polyhedral relaxation. For large instances, the exponential size of this family actually ends up hurting the solving time, by approximately 1\%.

In the same vein, one could improve this strategy as part of a branch \& cut algorithm by only considering, when building $Y$, the schedules and starting dates which have not been explicitly zeroed out by the branching so far. This yields another local cut, valid only for the subtree rooted at that node, but with a tighter upper bound on the right hand side.

\subsection{Professor complement cuts}

Suppose a course $c$ schedules a class $l$ on day $d$, and another course $c'$ schedules a clas $l'$ on day $d$. Then it better be the case that for any professor $p$, $p$ does \emph{not} teach both of these classes. In other words, that \emph{someone else} teaches those classes.

And yet, consider the following scenario:

\begin{itemize}
\item Two courses, $1$ and $2$.
\item A single role, $1$.
\item A single starting date, $1$.
\item Two classes for each, $1$ and $2$. Each class requires 1 professor with role $1$.
\item Two weekly schedules, $1$ and $2$. Schedule $1$ says class $1$ is on day $1$, class $2$ is on day $3$. Schedule $2$ says class $1$ is on day $2$, class $2$ is on day $3$.
\item A single professor, $1$, available all $3$ days and able to fill role $1$.
\end{itemize}

The schedule structure tells us that $ds_{1, 1, 1, 1} = 1$, $ds_{1, 1, 2, 3} = 1$, $ds_{2, 1, 1, 2} = 1$, and $ds_{2, 1, 2, 3} = 1$. Note that \emph{both} weekly schedules place the $2$nd class on day $3$.

Now consider the following vertex:

\begin{itemize}
\item $cp_{1, 1} = 0.5$
\item $cp_{1, 2} = 0.5$
\item $cp_{2, 1} = 0.5$
\item $cp_{2, 2} = 0.5$
\item $class\_date_{1, 2, 4} = 0.5$
\item $class\_date_{2, 2, 4} = 0.5$
\item $class\_date_{1, 2, 3} = 0.5$
\item $class\_date_{2, 2, 3} = 0.5$
\item $busy_{1, 1, 3} = 0.5$
\item $busy_{1, 1, 4} = 0.5$
\item $busy_{1, 2, 3} = 0.5$
\item $busy_{1, 2, 4} = 0.5$
\item $x_{1, 2, 1, 1} = 1$
\item $x_{2, 2, 1, 1} = 1$
\end{itemize}

So here we have a situation where classes $2$ of both courses are \emph{both} scheduled (in all integral vertices) on the same day ($3$), and yet the same professor ($1$) teaches both of them!

We can fix this by requiring that, since \emph{all} valid choices of a schedule and start date for courses $c = 1$ and $c' = 2$ result in classes $l = 2$ and $l' = 2$ being scheduled on the same day $d = 3$, then for every professor $p$, the following must hold:

$$
busy\_l_{p, c, l} + busy\_l_{p, c', l'} \le 1
$$

In this case, remembering that since we only have one role $busy\_l_{p, c, l} = x_{c, l, p, 1}$, instantiating the variables $c = 1$, $c' = 2$, $l = 2$, $l' = 2$, we obtain the constraints

\begin{align*}
&busy\_l_{1, 1, 2}& &+& &busy\_l_{1, 2, 2}& &\le& &1&\\
&x_{1, 2, 1, 1}& &+& &x_{2, 2, 1, 1}& &\le& &1&\\
&1& &+& &1& &\le& &1&
\end{align*}

Which means this vertex violates the cut.

Formally, the "professor complement" family is defined as:

Let
\begin{alignat*}{4}
Match(l, l', c, c') = &l \le n(c)\\
                      \land &l' \le n(c') \\
                      \land &\exists d \in D \mid &&\forall (p, p') \in M_c \times M_{c'},\\
                      &                           &&(w, w') \in W_c \times W_{c'},\\
                      &                           &&ds_{p, w, l, d} = 1 \land ds_{p', w', l', d} = 1)
\end{alignat*}

Then

\begin{align*}
&\forall (l, l') \in N^2\\
&\forall (c, c') \in C^2 \mid Match(l, l', c, c')\\
&\forall p \in P,\\
&busy\_l_{p, c, l} + busy\_l_{p, c', l'} \le 1
\end{align*}

