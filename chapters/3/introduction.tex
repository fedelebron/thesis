\section{Introduction}
The problem is, in its full generality, in \npc, as we showed in the previous chapter. Thus it is not likely that we will easily find a polynomial algorithm that will solve it. In attempting to find useful practical solution to the general case, we present an integer linear programming formulation, or ILP for short. The aim of this formulation is to leverage the power of modern day ILP solvers to quickly find solutions to practical scenarios.

In an ILP formulation we model the problem we are trying to solve as something of the following sort:

\begin{alignat*}{2}
\text{maximize } & \langle c, x \rangle\\
\text{subject to } & A x \le b\\
                   & x \ge 0\\
                   & x \in \Z^n
\end{alignat*}
for some $b, c \in \Z^n$, and $A \in \Z^{n \times n}$.

To every instance $I$ of the timetabling problem, we can asosciate a polytope $P(I)$, which is the convex hull of the integer solutions to the linear program we defined above. We can also consider the polytope formed by the solutions to the relaxation of the above linear program, that is, where we remove the requirement that the solution vector $x$ is integral. We will call this polytope $P_{\R}(I)$. Clearly  $P(I) \subset P_{\R}(I)$, since any convex combination of integral points is, in particular, a convex combination of real points. In the following sections we will analyze both polytopes. The reason we do this is because even though we are maximizing our objective function over $P(I)$, ILP solvers typically use geometric information from $P_{\R}(I)$ in order to solve the problem over $P(I)$, which has a more combinatoric structure. We will also develop ways to help these solvers to "trim" $P_{\R}(I)$ into $P(I)$, thus reducing the search space of solutions. For more information on ILP, see Appendix \ref{app:ilp}.
