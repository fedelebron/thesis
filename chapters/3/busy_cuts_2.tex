\subsection{Busy cuts}

Consider the following scenario:
\begin{itemize}
\item A single course, 1.
\item A single class, 1.
\item Two days, 1 and 2.
\item Two professors, available both days.
\item A single role, 1, which both professors can take.
\item A single starting week, 1.
\item Two weekly patterns, $p1$ and $p2$. $p1$ schedules class $1$ on day $1$, $p2$ schedules class $1$ on day $2$.
\end{itemize}

Then we could have the following fractional vertex:

\begin{itemize}
\item $csd_{1, 1} = 1$, remembering $csd_{c, sd}$ means course $c$ chooses starting week $sd$
\item $cp_{1, 1} = 0.5$, remembering $cp_{c, p}$ means course $c$ chooses schedule $p$
\item $cp_{1, 2} = 0.5$
\item $class\_date_{1, 1, 1} = 0.5$, remembering $class\_date_{c, l, d}$ means the $l$th class of course $c$ is scheduled on day $d$
\item $class\_date_{1, 1, 2} = 0.5$
\item $busy_{1, 1, 1} = 0$, remembering $busy_{p, c, d}$ means professor $p$ teaches a class for course $c$ on day $d$
\item $busy_{1, 1, 2} = 0$
\item $busy_{2, 1, 1} = 0$
\item $busy_{2, 1, 2} = 0$
\item $x_{1, 1, 1, 1} = 0.5$, remembering $x_{c, l, p, k}$ means professor $p$ teaches the $l$th class of course $c$ with role $k$
\item $x_{1, 1, 2, 1} = 0.5$
\end{itemize}

It may seem odd that we have class $1$ falling on days $1$ or $2$ (as per $class\_date$), and we've assigned professors $1$ and $2$ to teach the $1$st class, and yet \emph{neither} of them are busy on days $1$ or $2$!

But this makes sense given the constraints. Remember the way we lower bound $busy$:

$$
busy_{p, c, d} \ge busy\_l_{p, c, l} + class\_date_{c, l, d} - 1
$$

where $busy\_l_{p, c, l} = \sum_{k \in R} x_{c, l, p, k}$. In our case, with $d = 1$ or $d = 2$, we get $busy\_l_{1, 1, 1} = 0.5$, and $class\_date_{1, 1, d} = 0.5$. Thus the constraint is the same as

$$
busy_{1, 1, d} \ge 0
$$

which is trivial.

The problen here is that since each professor only works "half" the class, the class requirements are satisfied, but not the $busy$ requirements for each of the professors (at least conceptually - clearly the linear constraints themselves are satisfied!). 

We eliminate these vertices by noting that if a professor $p$ teaches the $l$th class of course $c$ with some role, and class $l$ can only fall on a subset $X$ of days, then it must hold that

$$
busy\_l_{p, c, l} \le \sum_{d \in X} busy_{p, c, d}
$$

Furthermore, we can always compute at least one such subset $X$: The set $d$ such that there's at least one valid weekly schedule and one valid starting week, such that the $l$th class falls on the $d$th day. This results in the following valid cutting plane:

\begin{align*}
&\forall c \in C\\
&\forall p \in P\\
&\forall 1 \le l \le n(c)\\
&busy\_l_{p, c, l} \ge \sum_{\substack{d \in D \\ s \in M\ \mid\ pp_{c, s} = 1\\ w \in W\ \mid\ psd_{c, w} = 1 \\ ds_{s, w, l, d} = 1}} busy_{p, c, d}
\end{align*}

The above fractional solution is cut, since this inequality says, for $c = 1, p = 1, l = 1$:

$$
busy\_l_{1, 1, 1} = \sum_{k \in R} x_{1, 1, 1, k} = x_{1, 1, 1, 1} \le busy_{1, 1, 1} + busy_{1, 1, 2}
$$

Where the right hand side is $0$, and the left hand side is $0.5$, rendering this solution invalid.

In practice, we see this family of cuts helps in a branch \& cut algorithm used in CPLEX and SCIP, lowering the solving time by about 10\% on large instances. As an improvement for a branch \& cut strategy, when one wishes to add a cut to a node in the branch \& cut tree, one can drop the terms corresponding to the schedules and starting days that haven't been explicitly zeroed out in the current node by the branching, tightening the bound. We note that this cut is a \emph{local} cut, since it need not be valid for other nodes in the tree, which might have chosen a different weekly schedule and starting date, and thus we cannot remove those terms from the summation.
