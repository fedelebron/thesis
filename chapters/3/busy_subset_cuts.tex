\subsection{Busy subset cuts}
There is a particularly interesting extension of the previous family of cuts. Consider the following scenario:
\begin{itemize}
\item A single course, $1$.
\item The course has two classes, $1$ and $2$.
\item There's only one role, $1$ and each class requires $1$ professor with that role.
\item Two available weekly schedule, $1$ and $2$. Schedule $1$ places class $1$ on day $1$, and class $2$ on day $2$. Schedule $2$ places class $1$ on day $2$, and class $2$ on day $3$.
\item A single starting date, $1$.
\item Two professors, $1$ and $2$, available both days, and able to fill role $1$.
\end{itemize}

Thus we have $ds_{1, 1, 1, 1} = 1$, $ds_{1, 1, 2, 2} = 1$, $ds_{2, 1, 1, 2} = 1$, and $ds_{2, 1, 2, 3} = 1$, remembering that $ds_{p, sd, l, d} = 1$ means schedule $p$ with starting date $sd$ has the $l$th class fall on day $d$.

Thus we see that classes $\{1, 2\}$ must fall in either days $\{1, 2\}$ or $\{2, 3\}$. Consider, however, the following vertex:

\begin{itemize}
\item $cp_{1, 1} = 0.5$
\item $cp_{1, 2} = 0.5$
\item $class\_date_{1, 1, 1} = 0.5$
\item $class\_date_{1, 1, 2} = 0.5$
\item $class\_date_{1, 2, 2} = 0.5$
\item $class\_date_{1, 2, 3} = 0.5$
\item $x_{1, 1, 3, 1} = 0.5$
\item $x_{1, 1, 1, 1} = 0.5$
%\item $x_{1, 1, 2, 1} = 0.5$
\item $x_{1, 2, 1, 1} = 0.5$
\item $x_{1, 2, 2, 1} = 0.5$
%\item $busy_{1, 1, 1} = 0.5$
\item $busy_{1, 1, 2} = 0.5$
\item $busy_{2, 1, 2} = 0.5$
\item $busy_{3, 1, 1} = 0.5$
%\item $busy_{2, 1, 1} = 0$
%\item $busy_{2, 1, 3} = 0$
\end{itemize}

where all other variables are $0$. One may wonder how it is possible that $busy_{2, 1, 3} = busy_{1, 1, 3} =  0$, when $class\_date_{1, 2, 3} = 0.5$, and $x_{1, 2, 1, 1} = x_{1, 2, 2, 1} = 0.5$. That is, they (professors $1$ and $2$) are both "semi-responsible" for class $2$, and class $2$ semi-falls on day $3$, yet \emph{neither} of them works on day $3$!

The model's restrictions aren't violated in this fractional solution, as this is our lower bound constraint on $busy$:

$$
busy_{p, c, d} \ge busy\_l_{p, c, l} + class\_date_{c, l, d} - 1
$$

for $p \in P$, $c \in C$, $1 \le l \le n(c)$, and $d \in D$.

Instantiated at $p \in \{1, 2\},$ $c = 1$, $l \in \{1, 2\}$, $d = 3$, and remembering that since we have only one role, $busy\_l_{p, c, l} = x_{c, l, p, 1}$, the model's busy lower bound constraints say:

\begin{itemize}
\item $busy_{1, 1, 1} \ge x_{1, 1, 1, 1} + class\_date_{1, 1, 3} - 1 \iff 0 \ge 0.5 + 0 - 1$ \greencheck
\item $busy_{1, 1, 2} \ge x_{1, 2, 1, 1} + class\_date_{1, 2, 3} - 1 \iff 0.5 \ge 0.5 + 0.5 - 1$ \greencheck
\item $busy_{2, 1, 1} \ge x_{1, 1, 2, 1} + class\_date_{1, 1, 3} - 1 \iff 0 \ge 0 + 0 - 1$ \greencheck
\item $busy_{2, 1, 2} \ge x_{1, 2, 2, 1} + class\_date_{1, 2, 3} - 1 \iff 0.5 \ge 0.5 + 0.5 - 1$ \greencheck
\end{itemize}

The previous family of cuts, the busy cuts, doesn't separate this vertex either, since we do have 
\begin{itemize}
\item $x_{1, 1, 1, 1} \le busy_{1, 1, 1} + busy_{1, 1, 2}$ \greencheck
\item $x_{1, 2, 1, 1} \le busy_{1, 1, 2} + busy_{1, 1, 3}$ \greencheck
\item $x_{1, 1, 2, 1} \le busy_{2, 1, 1} + busy_{2, 1, 2}$ \greencheck
\item $x_{1, 2, 2, 1} \le busy_{2, 1, 2} + busy_{2, 1, 3}$ \greencheck
\end{itemize}

What is missing, however, is that if professor $p$ works on a set $X$ of classes for course $c$, and the set of possible dates for those classes (i.e. the union, for all possible choices of schedule and starting date, of the days the classes in $X$ fall) is called $Y$, then the sum of the $busy\_l_{p, c, x}$ for $x \in X$ \emph{must} be smaller than or equal to the sum of the days, out of $Y$, that $p$ works on course $c$, because for every class $p$ works on, he works on exactly one day for that course, and we sum over all possible days these classes could fall in. Formally:

\begin{align*}
&\forall c \in C,\\
&\forall p \in P,\\
&\forall X \subset \{1 .. n(c)\}\\
&\text{Let }Y = \{d \in D \mid \exists l \in X, p \in M, w \in W \mid ds_{p, w, l, d} = 1\}\\
&\text{Then, }\\
&\sum_{l \in X} busy\_l_{p, c, l} \le \sum_{d \in Y} busy_{p, c, d}
\end{align*}

If we instantiate this family with $c = 1, p = 1, X = \{1, 2\}$, then we have $Y = \{1, 2, 3\}$, and the inequality

\begin{align*}
&x_{1, 1, 1, 1}& &+& &x_{1, 2, 1, 1}& &\le& &busy_{1, 1, 1}& &+& &busy_{1, 1, 2}& &+& &busy_{1, 1, 3}&\\
&0.5& &+& &0.5& &\le& &0& &+& &0.5& &+& &0&
\end{align*}

So this fractional solution violates this cut. We call this family of cuts the "busy subset" cuts. Conceptually, they are stronger than sums of the previous "busy" cuts, since if two pairs (weekly schedule, starting date) place two classes in $X$ on the same day $d$, these cuts only count that day once in the right hand side (since $d$ will only be counted once in $Y$), whereas summing over all the busy cuts would count them twice. Thus this is a tighter upper bound on $busy\_l$.

In practice, we observe that for small and medium sized instances, adding this family of cuts yields integral solutions for the polyhedral relaxation. For large instances, the exponential size of this family actually ends up hurting the solving time, by approximately 1\%.

In the same vein, one could improve this strategy as part of a branch \& cut algorithm by only considering, when building $Y$, the schedules and starting dates which have not been explicitly zeroed out by the branching so far. This yields another local cut, valid only for the subtree rooted at that node, but with a tighter upper bound on the right hand side.
