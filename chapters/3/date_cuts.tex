\subsection{Class date cuts}

It is often the case that the instance data implies that no two classes of the same course are to be scheduled on the same day. In this case, we have a simple family of cuts.

Let us start with an example of a situation where an undesirable fractional solution appears. Consider the following scenario:

\begin{itemize}
  \item Schedule 1: One class per week, on mondays.
  \item Schedule 2: Three classes per week, on mondays, wednesdays, and fridays.
\end{itemize}

Course $c$ has 3 classes, and the semester lasts one week (to simplify the scenario). It also has a single possible starting week, the first one. Assume there are enough professors available to fill all role requirements of course $c$. In this case, the only possible schedule $c$ can pick is schedule 2, since picking schedule $1$ would result in not scheduling two of $c$'s classes.

Now consider the linear relaxation of this scenario. It is perfectly possible to have $cp_{c, 1} = 0.5$, and $cp_{c, 2} = 0.5$. Considering the start week as $1$ (thus $csd_{c, 1} = 1$), the first monday as $1$, the problem constraints mean that

$$
cp_{c, 1} + csd_{c, 1} - 1 \le class\_date_{c, 1, 1}
$$

Since we said there was only one starting week, $csd_{c, 1} = 1$. We also said $cp_{c, 1} = 0.5$. Thus our constraint becomes

$$
0.5 \le class\_date_{c, 1, 1}
$$

And we also said schedule $2$, with starting week $1$,  placed the $1$st class on day $1$, which also yields

$$
0.5 \le class\_date_{_c, 1, 1}
$$

Now since every class must be allocated to some day, we have a constraint $\sum_{d \in D} class\_date_{c, 1, d} = 1$. So let's say our solution vector places $class\_date_{c, 1, 1} = 1$.

Since schedule $1$ has no day allocated for class $3$, the following inequality must be true by \ref{eq:coclassdate}

$$
(1 - cp_{c, 1}) + (1 - csd_{c, 1}) \ge class\_date_{c, 3, 1}
$$

Since $ds_{1, 1, 3, 1} = 0$. That is, picking the schedule $1$ and starting week $1$, the $3$rd class of a course does \emph{not} fall on the $1$st day. However, since we have a fractional $cp_{c, 1} = 0.5$, and $csd_{c, 1} = 1$, this becomes

$$
0.5 \ge class\_date_{c, 3, 1}
$$

So we can end up with a solution vector where $class\_date_{c, 3, 1} = 0.5$, and $class\_date_{c, 1, 1} = 1$. But then, if we sum for all $l$ the value of $class\_date_{c, l, 1}$ we get a value exceeding $1$, which is impossible since we must at most have one class per day for this course (in particular, at most one class on day $1$).

Thus we derive the following cut:

\begin{align}
&\forall c \in C,\\
&\forall d \in D,\\
&\sum_{l = 1}^{n(c)} class\_date_{c, l, d} \le 1
\end{align}

Thus rendering this fractional solution invalid, but still being a valid inequality for all integral solutions.
