\documentclass[11pt]{amsart}
\usepackage[utf8]{inputenc}
\usepackage{geometry}                % See geometry.pdf to learn the layout options. There are lots.
\geometry{letterpaper}                   % ... or a4paper or a5paper or ... 
%\geometry{landscape}                % Activate for for rotated page geometry
%\usepackage[parfill]{parskip}    % Activate to begin paragraphs with an empty line rather than an indent
\usepackage{graphicx}
\usepackage{amssymb}
\usepackage{epstopdf}
\DeclareGraphicsRule{.tif}{png}{.png}{`convert #1 `dirname #1`/`basename #1 .tif`.png}

\title{Planteo del problema}
\author{Federico Lebrón}
%\date{}                                           % Activate to display a given date or no date

\begin{document}
\maketitle
%\section{}
%\subsection{}


Variables:
\begin{enumerate}
\item $assigned[c][l][p][t] = $ El profesor $p$ está asignado con el rol $t$ a la $l$-ava clase del $c$-avo curso.
\item $offset[c][i] = $ Hay que atrasar el comienzo del $c$-avo curso $i$ días.
\item $pattern[c][i] = $ El patrón semanal del $c$-avo curso es el $i$-ésimo.
\end{enumerate}

Constantes:

\begin{enumerate}
\item $allowed\_pattern[c][i] = $ El $c$-avo curso puede usar el $i$-ésimo patrón.
\item $pattern\_day[p][l][d] = $ El $p$-avo patrón dice que la $l$-ava clase se cursa el $d$-avo día.
\item $requirements[c][l][t] = $ Cuántos profesores del tipo $t$ se necesitan para la $l$-ava clase del $c$-avo curso.
\item $can\_act\_as[p][t] = $ El profesor $p$ puede actuar como el tipo $t$.
\item $available[p][d] = $ El profesor $p$ está disponible el día $d$.
\end{enumerate}

Restricciones:

\begin{enumerate}
\item Si el profesor $p$ está asignado a una clase con el rol $t$, entonces debe poder actuar con el rol $t$.
$$
\forall p \in P, t \in T, c \in C, l \in L, assigned[c][l][p][t] < can\_act\_as[p][t]
$$

\item Si el profesor $p$ está asignado a la $l$-ava clase del curso $c$, y el curso $c$ tiene el patrón $r$, y empieza $i$ días tarde, y $pattern\_day[r][l][d]$ es 1, entonces $p$ tiene que estar disponible en el día $d + i$.
\end{enumerate}

\end{document}
