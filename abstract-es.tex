Los problemas de asignación de horarios han sido ampliamente estudiados en la literatura de operations research. Esta importante parte del planeamiento de cursos es conocida por ser particularmente difícil, ambas en la teoría y en la práctica, en varios casos importantes. Es por esta razón que es estudiado por la comunidad de la optimización combinatoria. En parituclar, es estudiado usando modelos de programación lineal entera.

In este trabajo consideramos una versión de este problema y la estudiamos desde tanto el punto de vista teórico como el práctico. Llamamos a esta variación el Problema de Asignación de Cursos y Profesores (CSPAP por su sigla en inglés). Las restricciones que consideramos son las de disponibilidad de los profesores, roles que pueden ocupar, variabilidad de fechas de inicio de los cursos, y variabilidad de patrones semanales de los cursos. Estudiamos el cambio de fase en la complejidad computacional cuando algunas de estas restricciones son aplicadas o levantadas, y vemos cómo el problema cambia de pertenecer a la clase $\p$ a la clase $\npc$.

Para el problema general, consideramos un enfoque polihedral, que produce resultados eficientes para problemas de tamaños prácticos. Para instancias de tamaños más grandes, el rendimiento decae. Por este motivo desarrollamos familias de desigualdades lineales, aumentando la eficiencia de las estrategias branch-and-bound de programación lineal entera. Junto con estas desigualdades, problemas de tamaño práctico son resueltos en cuestión de segundos.
