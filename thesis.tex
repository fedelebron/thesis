\documentclass[a4paper,11pt]{amsbook}
%\usepackage[utf8]{inputenc}

\usepackage{etoolbox}
\usepackage{chngcntr}
\usepackage{amsmath, amscd, amssymb, amsthm, latexsym}
\usepackage[usenames,dvipsnames,table]{xcolor}
\usepackage[spanish,english]{babel}
\usepackage [autostyle, english = american]{csquotes}
\MakeOuterQuote{"}
\usepackage{graphicx}
\usepackage{cite}
\usepackage{fontspec}
\setmainfont[SmallCapsFont = Fontin SmallCaps]{Fontin}
\usepackage{tikz}
\usetikzlibrary{calc}
\usepackage{hyperref}
\usepackage{clrscode3e}
\usepackage{import}
\usepackage{minted}
%\usepackage{fullpage}
% \usepackage{clrscode3e}	--algun paquete de algoritmos
\newtheorem{defn}{Definition}
\newtheorem{prop}{Proposition}
\newtheorem{thm}{Theorem}
\newtheorem{lem}{Lemma}
\newtheorem{cor}{Corollary}
\newtheorem{obs}{Observation}
\newtheorem{eg}{Example}

%For \autoref to add the word before number
\newcommand*{\defnautorefname}{Definition}
\newcommand*{\propautorefname}{Proposition}
\newcommand*{\thmautorefname}{Theorem}
\newcommand*{\lemautorefname}{Lemma}
\newcommand*{\corautorefname}{Corollary}
\newcommand*{\obsautorefname}{Observation}
\newcommand*{\egautorefname}{Example}
\def\sectionautorefname{Section}
\def\chapterautorefname{Chapter}

\tikzset{
  v/.style={draw, circle, inner sep=0pt, font=\small, align=center, minimum size=0.75cm},
  box/.style={draw, rounded corners=3pt},
  e/.style={->, dashed}
}
\newcommand{\dottedarrow}[2]{\draw[e] ($(#1)!0.5cm!(#2)$) -> ($(#1)!1cm!(#2)$)}
\newcommand{\dottedarrows}[4]{
  \foreach \i in #1 {
    \foreach \j in #2 {
      \ifnum\pdfstrcmp{\i}{#3}=0
        \ifnum\pdfstrcmp{\j}{#4}=0
        \else
          \dottedarrow{\i}{\j};
        \fi
      \else
        \dottedarrow{\i}{\j};
      \fi
    }
  }
}

\def\checkmark{\tikz\fill[scale=0.4](0,.35) -- (.25,0) -- (1,.7) -- (.25,.15) -- cycle;}
\def\scalecheck{\resizebox{\widthof{\checkmark}*\ratio{\widthof{x}}{\widthof{\normalsize x}}}{!}{\checkmark}}
\def\greencheck{{\color{green}\checkmark}}
\newlength{\problemoffset}
\setlength{\problemoffset}{0.5in}
\newcommand{\decision}[3]{%     \decision{NAME}{INSTANCE}{QUESTION}
\begin{list}{}{
\setlength{\leftmargin}{\problemoffset}
\setlength{\rightmargin}{\problemoffset}
\setlength{\parsep}{0pt}
\setlength{\itemsep}{2pt}
\setlength{\topsep}{\itemsep}
\setlength{\partopsep}{\itemsep}
}
\item
{\textsc{#1}}
\item
{INSTANCE: #2}
\item
{QUESTION: #3}
\end{list}
}

\definecolor{light-gray}{gray}{0.75}

\newcommand{\bigoh}{\ensuremath{\mathcal{O} }}
\newcommand{\N}{\mathbb{N}}
\newcommand{\Nz}{\mathbb{N}_0}
\newcommand{\Z}{\mathbb{Z}}
\newcommand{\Q}{\mathbb{Q}}
\newcommand{\R}{\mathbb{R}}
%\newcommand{\G}{\mathcal{G}}
\newcommand{\Parts}{\mathcal{P}}
\newcommand{\dist}{{\rm dist}}
\newcommand{\adj}{{\rm Adj}}
\newcommand{\symdiff}{\bigtriangleup}
\newcommand{\npc}{\ensuremath{\mathcal{NP}{\text{-complete}}} }
\newcommand{\npces}{\ensuremath{\mathcal{NP}{\text{-completo}}} }
\newcommand{\nph}{\ensuremath{\mathcal{NP}{\text{-hard}}} }
\newcommand{\np}{\ensuremath{\mathcal{NP}} }
\newcommand{\p}{\ensuremath{\mathcal{P}} }
\newcommand{\gic}{\ensuremath{\mathcal{GI}{\text{-complete}}} }
\newcommand{\gih}{\ensuremath{\mathcal{GI}{\text{-hard}}} }
\newcommand{\gi}{\ensuremath{\mathcal{GI}} }
\newcommand{\mcesp}{{\textsc{MCESP} }}
\newcommand{\mincesp}{{\textsc{MinCESP} }}
\newcommand{\maxcut}{{\textsc{Max-Cut} }}
\newcommand{\yes}{{\textsc{Yes}} }
\newcommand{\no}{{\textsc{No}} }
\DeclareMathOperator{\tr}{tr}
\newcommand{\todo}[1]{\textcolor{red}{TODO : #1}}
\newcommand{\note}[1]{\textcolor{blue}{NOTA : #1}}
\newcommand{\HRule}{\rule{\linewidth}{0.5mm}}
\renewcommand\qedsymbol{\ensuremath{\blacksquare}}
\renewcommand*{\subsectionautorefname}{Subsection}
\newcommand\restr[2]{{% we make the whole thing an ordinary symbol
  \left.\kern-\nulldelimiterspace % automatically resize the bar with \right
  #1 % the function
  \vphantom{\big|} % pretend it's a little taller at normal size
  \right|_{#2} % this is the delimiter
  }}


\hypersetup{
linkcolor = blue,
colorlinks = true,
citecolor = magenta
}

\DeclareRobustCommand{\gobblefive}[5]{}
\newcommand*{\SkipTocEntry}{\addtocontents{toc}{\gobblefive}}

\newcounter{aligncounter}
\counterwithin*{aligncounter}{section}
\counterwithin*{equation}{aligncounter}
\AtBeginEnvironment{align}{\stepcounter{aligncounter}}
\AtBeginEnvironment{alignat}{\stepcounter{aligncounter}}

\begin{document}
%\numberwithin{section}{chapter}
%\setcounter{section}{1}
%\numberwithin{subsection}{section}
\numberwithin{prop}{section}
%\numberwithin{prop}{subsection}
\numberwithin{lem}{section}
%\numberwithin{lem}{subsection}
\numberwithin{thm}{section}
%\numberwithin{thm}{subsection}
\numberwithin{cor}{section}
%\numberwithin{cor}{subsection}
\numberwithin{defn}{section}
%\numberwithin{defn}{subsection}
\numberwithin{equation}{section} %sets equation numbers <chapter>.<section>.<index>
%\numberwithin{equation}{subsection} %sets equation numbers <chapter>.<section>.<subsection>.<index>
%\numberwithin{equation}{subsubsection} %sets equation numbers <chapter>.<section>.<subsection>.<subsubsection>.<index>
\numberwithin{obs}{section} %sets equation numbers <chapter>.<section>.<index>
%\numberwithin{obs}{subsection} %sets equation numbers <chapter>.<section>.<subsection>.<index>
\numberwithin{figure}{section} %sets equation numbers <chapter>.<section>.<index>
%\numberwithin{figure}{subsection} %sets equation numbers <chapter>.<section>.<subsection>.<index>
%\maketitle

\renewcommand\theequation{\thesection.\thealigncounter.\arabic{equation}}


\begin{titlepage}

\begin{center}


% Upper part of the page
\begin{flushleft}
\raisebox{-.5\height}{\includegraphics[width=13em]{./graphics/logofcen}}%
\hfill
\raisebox{-.5\height}{\includegraphics[width=15em]{./graphics/dc_logo}}\\[2.0cm]
\end{flushleft}

{\large \sc Universidad de Buenos Aires

Facultad de Ciencias Exactas y Naturales

Departamento de Computaci\'on} \\[2cm]

% Title
%\HRule \\[0.4cm]
{ \LARGE \bfseries A study on the computational complexity of timetabling problems}\\[2.6cm]
%\HRule \\[1.2cm]
{\large Tesis presentada para optar al t\'{\i}tulo de\\
Licenciado en Ciencias de la Computaci\'on}
\vfill
% Author and supervisor
{\large \emph{Autor:} Federico \textsc{Lebrón}\\[0.5cm]
\emph{Director:} 
Dr.~Javier Leonardo \textsc{Marenco} \\[0.5cm]
\emph{Jurados:} 
Dr.~Min Chih \textsc{Lin} y 
Dra.~Isabel \textsc{Méndez-Díaz}
}
\vfill

{\large \today}

\end{center}
\end{titlepage}

\setcounter{tocdepth}{3} %temporal

\pagenumbering{roman}
%\SkipTocEntry \chapter*{Abstract}
%Los problemas de asignación de horarios han sido ampliamente estudiados en la literatura de investigación operativa. Esta importante parte del planeamiento de cursos es conocida por ser particularmente difícil, ambas en la teoría y en la práctica, en varios casos importantes. Es por esta razón que es estudiado por la comunidad de la optimización combinatoria. En particular, es estudiado usando modelos de programación lineal entera.

En este trabajo consideramos una versión de este problema y la estudiamos tanto desde el punto de vista teórico como el práctico. Llamamos a esta variación el Problema de Asignación de Cursos y Profesores (CSPAP por su sigla en inglés). Las restricciones que consideramos son las de disponibilidad de los profesores, roles que pueden ocupar, variabilidad de fechas de inicio de los cursos, y variabilidad de patrones semanales de los cursos. Estudiamos el cambio de fase en la complejidad computacional cuando algunas de estas restricciones son aplicadas o levantadas, y vemos cómo el problema pasa de pertenecer a la clase $\p$ a la clase $\npces$.

Para el problema general, consideramos un enfoque poliedral, que produce resultados eficientes para problemas de tamaños prácticos. Para instancias de tamaños más grandes, el rendimiento decae. Por este motivo desarrollamos familias de desigualdades válidas, aumentando la eficiencia de los algoritmos de tipo branch-and-bound de programación lineal entera. Al utilizar estas desigualdades, problemas de tamaño práctico son resueltos en cuestión de segundos.

\SkipTocEntry \chapter*{Abstract}
Timetabling problems are a widely studied topic in operations research literature. This important part of course planning is known to be exceedingly hard, both in theory and in practice, in several important cases. For this reason it has been studied by the combinatorial optimization community. In particular, it's been studied using integer linear programming models.

In this work we consider one version of the problem and proceed to study it, both from a theoretical and practical standpoint. We name this variation the Course Scheduling and Professor Assignment Problem (CSPAP). The restrictions we consider are ones of availability of faculty, faculty roles, course start date, and weekly course pattern. We study the complexity phase change that occurs when some restrictions are lifted or applied, and see how the problem shifts from $\p$ to $\npc$.

For the general problem, we consider a polyhedral approach, which yields good performance for practical problem sizes. For large instances however, we see that the performance decays. In order to help in these situations, we find families of valid linear inequalities to increase the efficiency of a branch-and-bound integer linear programming solver. Together with these inequalities, solutions to practical problems can be found in a matter of seconds.

\let\cleardoublepage\clearpage
\SkipTocEntry \chapter*{Abstract}
Los problemas de asignación de horarios han sido ampliamente estudiados en la literatura de investigación operativa. Esta importante parte del planeamiento de cursos es conocida por ser particularmente difícil, ambas en la teoría y en la práctica, en varios casos importantes. Es por esta razón que es estudiado por la comunidad de la optimización combinatoria. En particular, es estudiado usando modelos de programación lineal entera.

En este trabajo consideramos una versión de este problema y la estudiamos tanto desde el punto de vista teórico como el práctico. Llamamos a esta variación el Problema de Asignación de Cursos y Profesores (CSPAP por su sigla en inglés). Las restricciones que consideramos son las de disponibilidad de los profesores, roles que pueden ocupar, variabilidad de fechas de inicio de los cursos, y variabilidad de patrones semanales de los cursos. Estudiamos el cambio de fase en la complejidad computacional cuando algunas de estas restricciones son aplicadas o levantadas, y vemos cómo el problema pasa de pertenecer a la clase $\p$ a la clase $\npces$.

Para el problema general, consideramos un enfoque poliedral, que produce resultados eficientes para problemas de tamaños prácticos. Para instancias de tamaños más grandes, el rendimiento decae. Por este motivo desarrollamos familias de desigualdades válidas, aumentando la eficiencia de los algoritmos de tipo branch-and-bound de programación lineal entera. Al utilizar estas desigualdades, problemas de tamaño práctico son resueltos en cuestión de segundos.

%\SkipTocEntry \chapter*{Acknowledgments}
%blahblah


\tableofcontents
\newpage
\pagenumbering{arabic}
\subimport{chapters/1/}{main}
\subimport{chapters/2/}{main}
\subimport{chapters/3/}{main}
\subimport{chapters/4/}{main}
\subimport{chapters/5/}{main}
%\bibliography{references}{}
%\bibliographystyle{plain}

\nocite{*}
\bibliographystyle{plain}
\bibliography{bibliography}


\end{document}
