Timetabling problems are a widely studied topic in operations research literature. This important part of course planning is known to be exceedingly hard, both in theory and in practice, in several important cases. For this reason it has been studied by the combinatorial optimization community. In particular, it's been studied using integer linear programming models.

In this work we consider one version of the problem and proceed to study it, both from a theoretical and practical standpoint. We name this variation the Course Scheduling and Professor Assignment Problem (CSPAP). The restrictions we consider are ones of availability of faculty, faculty roles, course start date, and weekly course pattern. We study the complexity phase change that occurs when some restrictions are lifted or applied, and see how the problem shifts from $\p$ to $\npc$.

For the general problem, we consider a polyhedral approach, which yields good performance for practical problem sizes. For large instances however, we see that the performance decays. In order to help in these situations, we find families of valid linear inequalities to increase the efficiency of a branch-and-bound integer linear programming solver. Together with these inequalities, solutions to practical problems can be found in a matter of seconds.
